\documentclass{article}
\usepackage{../../Self_Style}

\title{Math CS 121 HW 3}
\author{Zih-Yu Hsieh}
\date{\today}

\begin{document}
\maketitle

\section*{1}
\begin{ques}\label{q1}
    3.3.18:

    Suppose that $10$ good and $3$ dead batteries are mixed up. Jack tests them one by one, at random and without replacement. But before testing the fifth battery he realizes that he does not remember whether the first one tested is good or is dead. All he remembers is that the last three that were tested were all good. What is the probability that the first one is also good?
\end{ques}

\begin{proof}
    Let $A$ denotes the event of the first battery being good, while $B$ denotes the event of the $2^\textmd{nd}$ to $4^\textmd{th}$ ones being good. Then, the scenario described has event $B$ being true, which serves as the condition. 

    Notice that $A \cap B$ denotes the event that the first four tested batteries are all good, which since there are total of $13$ batteries with $10$ being good (and $3$ being bad), $\PP(A\cap B) = \frac{\begin{pmatrix}10\\4\end{pmatrix}}{\begin{pmatrix}13\\4\end{pmatrix}}=\frac{42}{143}$ (since we need to choose $4$ good batteries out of the total of $10$ good ones, while there are toal of $13$ choose $4$ ways of doing so).

    Similarly, the probability of $B$ happening is given as follow (where we care about the second to the fourth ones):
    \begin{align}
        \PP(B)&=\PP(\textmd{First one being good}) \cdot \PP(\textmd{Next 2 to 4 being good})+\PP(\textmd{First one being bad})\cdot \PP(\textmd{Next 2 to 4 being good})\\
        &= \frac{10}{13}\cdot \frac{\begin{pmatrix}9\\3\end{pmatrix}}{\begin{pmatrix}12\\3\end{pmatrix} }+\frac{3}{13}\cdot\frac{\begin{pmatrix}10\\3\end{pmatrix} }{\begin{pmatrix}12\\3\end{pmatrix}} = \frac{60}{143}
    \end{align}
    (Note: After taking out the first battery without replacement, there are $12$ batteries left in each case. If the first one is good, only $9$ goods are left, while if the first one is bad, $10$ goods are left instead, that's why it's in the above form).

    Hence, the conditional probability $\PP(A|B)=\frac{\PP(A\cap B)}{\PP(B)}=\frac{42}{60}=\frac{7}{10}$.
\end{proof}

\newpage

\section*{2}
\begin{ques}\label{q2}
    3.3.19:

    A box contains $18$ tennis balls, of which eight are new. Suppose that three balls are selected randomly, played with, and after play are returned to the box. If other three balls are selected for play a second time, what is the probability that they are all new?
\end{ques}

\begin{proof}
    Given three balls are selected randomly initially, then there are four disjoint events: either all three are old (denoted as $A_0$), one is new and two are old (denoted as $A_1$), two are new and one is old (denoted as $A_2$), or all three are new (denoted as $A_3$). If we calculated their probability, we get (Note: there are initially $8$ new balls and $10$ old balls, total of $18$ balls. And here, we're choosing the balls without order):
    \begin{align}
        \PP(A_0)=\frac{\begin{pmatrix}10\\3\end{pmatrix}}{\begin{pmatrix}18\\3\end{pmatrix}}=\frac{5}{34}, \quad \PP(A_1)=\frac{\begin{pmatrix}10\\2\end{pmatrix}\begin{pmatrix}8\\1\end{pmatrix}}{\begin{pmatrix}18\\3\end{pmatrix}}=\frac{15}{34}, \quad \PP(A_2)=\frac{\begin{pmatrix}10\\1\end{pmatrix}\begin{pmatrix}8\\2\end{pmatrix}}{\begin{pmatrix}18\\3\end{pmatrix}}=\frac{35}{102}, \quad \PP(A_3)=\frac{\begin{pmatrix}8\\3\end{pmatrix}}{\begin{pmatrix}18\\3\end{pmatrix}}=\frac{7}{102}
    \end{align}
    Then, if $A_0$ happens, there are still $8$ new balls (since no new balls are selected); if $A_1$ happens, there are $7$ new balls left (since one new ball is selected and became an old one); if $A_2$ happens, there are $6$ new balls left (since two new balls are selected); similarly, if $A_3$ happens, there are only $5$ new balls left.

    So, let $W$ denotes the event that during the second selection (so, the selection has happened), all three balls newly selected are new, then it's probability is given as follow:
    \begin{align}
        \PP(W)&=\PP(W|A_0)\cdot \PP(A_0)+\PP(W|A_1)\cdot \PP(A_1)+\PP(W|A_2)\cdot \PP(A_2)+\PP(W|A_3)\cdot \PP(A_3)\\
        &= \frac{\begin{pmatrix}8\\3\end{pmatrix}}{\begin{pmatrix}18\\3\end{pmatrix}}\cdot \frac{5}{34}+\frac{\begin{pmatrix}7\\3\end{pmatrix}}{\begin{pmatrix}18\\3\end{pmatrix}}\cdot \frac{15}{34}+\frac{\begin{pmatrix}6\\3\end{pmatrix}}{\begin{pmatrix}18\\3\end{pmatrix}}\cdot\frac{35}{102}+\frac{\begin{pmatrix}5\\3\end{pmatrix}}{\begin{pmatrix}18\\3\end{pmatrix}}\cdot\frac{7}{102} = \frac{3185}{83232}\approx 0.0383
    \end{align}
    Here, $W$ is an intersection of ``all newly selected 3 balls are new'' and ``first selection of 3 balls had happened'', but since the second part is guaranteed to happen (or this becomes our new sample space), then the above is the desired probability (or there's no need to consider if the condition of $A_0\sqcup A_1\sqcup A_2\sqcup A_3$ had happened or not).

    (Note: The conditional probability is calculated based on the ``updated scenario'' and the corresponding number of new balls).
\end{proof}

\newpage
\section*{3}
\begin{ques}\label{q3}
    3.4.8:

    Urns I,II, and III contain three pennies and four dimes, two pennies and five dimes, three pennies and one dime, respectively. One coin is selected at random from each urn. If two of the three coins are dimes, what is the probability that the coin selected from urn I is a dime?
\end{ques}

\begin{proof}
    For this, since there are only one coin coming from each urn, while there are exactly two dimes and one penny eventually, hence it splits into three disjoint events: The penny is coming from I while the two dimes are from II and III (denoted as $A_I$), the penny is coming from II while the two dimes are from I and III (denoted as $A_{II}$), and the penny is coming from III while the two dimes are from I and III (denoted as $A_{III}$). As a result, they have the following probabilities(using the fact that we're choosing one from each urns, while I, II have 7 coins, and III has 4 coins):
    \begin{align}
        \PP(A_I)=\frac{3}{7}\cdot\frac{5}{7}\cdot\frac{1}{4}=\frac{15}{196},\quad \PP(A_{II})=\frac{4}{7}\cdot\frac{2}{7}\cdot\frac{1}{4}=\frac{8}{196},\quad \PP(A_{III})=\frac{4}{7}\cdot\frac{5}{7}\cdot\frac{3}{4}=\frac{60}{196}
    \end{align}
    Which, let $A$ denotes the event that eventually gets two dimes and one penny, we have $A=A_I\sqcup A_{II}\sqcup A_{III}$, hence $\PP(A)=\PP(A_I)+\PP(A_{II})+\PP(A_{III})=\frac{83}{196}$.

    \hfil

    Now, let $W$ denotes the event where the coin selected from urn I is a dime, this corresponds to event $A_{II},A_{III}$ if $A$ happens (while for $A_I$ urn I had gotten a penny, which is not possible to get a dime). Then, the probability of $W$ and $A$ happening together is given by:
    \begin{align}
        \PP(W \cap A)&=\PP(W|A_I)\cdot \PP(A_I)+\PP(W|A_{II})\cdot\PP(A_{II})+\PP(W|A_{III})\cdot\PP(A_{III})\\
        &= 0\cdot \frac{15}{196}+1\cdot\frac{8}{196}+1\cdot\frac{60}{196}=\frac{68}{196}
    \end{align}
    Hence, as a consequence the probability that $W$ happens given that $A$ happens, is $\PP(W|A)=\frac{\PP(W\cap A)}{\PP(A)} = \frac{68}{196}\cdot\frac{196}{83} = \frac{68}{83}$.
\end{proof}

\newpage

\section*{4}
\begin{ques}\label{q4}
    3.4.11:

    With probability of $1/6$ there are $i$ defective fuses aong 1000 fuses ($i=0,1,2,3,4,5$). If among $100$ fuses selected at random, none was defective, what is the probability of no defective fuses at all?
\end{ques}

\begin{proof}
    Let $A$ denotes the event that the $100$ selected fuses are all not defective, while let $B$ denotes the event where no fuses are defective. Then, to compute $\PP(B|A)$ here, using Bayes Rule we have $\PP(B|A)=\frac{\PP(A|B)\cdot\PP(B)}{\PP(A)}$.

    First, if given that $B$ happens (where there are no defective fuses), then $A$ would always happen (since the $100$ selected fuses will always be coming from all the non-defective fuses). So, $\PP(A|B)=1$.

    Then, the probability that $B$ happens (where there's no defective fuses, or $i=0$ for the notations given in the question) is given by $\PP(B)=\PP(\{i=0\}) = \frac{1}{6}$.

    Finally, the probability that $A$ happens (where the $100$ selected fuese are all non-defective), it can be classified into disjoint cases of having $i=0,1,...,5$ defective fuses. Since one of these events must happen (given that each $i$ is equally likely to happen with probability $1/6$, which the total probability of all cases of $i$ is $1$), by conditional probability, we get:
    \begin{align}
        \PP(A)=\sum_{j=0}^{5}\PP(A|\{i=j\})\cdot\PP(\{i=j\}) = \frac{1}{6}\sum_{j=0}^{5}\PP(A|\{i=j\})
    \end{align}
    Where, if there are $i=j$ number of defective fuses, then the probability of having all $100$ randomly selected fuses all being non-defective, is $\PP(A|\{i=j\})=\frac{\begin{pmatrix}1000-j\\100\end{pmatrix}}{\begin{pmatrix}1000\\100\end{pmatrix}}$. Hence, the above sum becomes:
    \begin{align}
        \PP(A)=\frac{1}{6}\sum_{j=0}^{5}\frac{\begin{pmatrix}1000-j\\100\end{pmatrix}}{\begin{pmatrix}1000\\100\end{pmatrix}} \approx 0.7807
    \end{align}
    Glue all the pieces together, we get:
    \begin{align}
        \PP(B|A)=\frac{\PP(A|B)\cdot\PP(B)}{\PP(A)} \approx \frac{1\cdot 1/6}{0.7807}\approx 0.2135
    \end{align}
\end{proof}

\newpage

\section*{5}
\begin{ques}\label{q5}
    3.5.21:

    Prove that if $A,B$, and $C$ are independent, then $A$ and $B \cup C$ are independent. Also show that $A\setminus B$ and $C$ are independent.
\end{ques}

\begin{proof}
    Suppose $A,B$, and $C$ are independent. Then we have $\PP(A\cap B)=\PP(A)\PP(B)$ (where the same logic applies to any two distinct pairs of events from $A,B,C$), and $\PP(A\cap B\cap C)=\PP(A)\PP(B)\PP(C)$.

    \hfil

    Now, if consider $A$ and $B\cup C$, since $B \cup C=B \sqcup (C\setminus B) = B\sqcup (C\setminus (B\cap C))$, then we have $\PP(B\cup C)=\PP(B)+\PP(C\setminus (B\cap C))=\PP(B)+\PP(C)-\PP(B\cap C)$. Also, since $A\cap (B\cup C) = (A\cap B)\sqcup (A\cap (C\setminus (B\cap C))) = (A\cap B)\sqcup ((A\cap C)\setminus (A\cap B\cap C))$ (since if $x \in A\cap (C\setminus(B\cap C))$, we have $x\in A,x\in C$, while $x\notin B$, so $x\in (A\cap C)\setminus(A\cap B\cap C)$; on the other hand, if $x\in (A\cap C)\setminus(A\cap B\cap C)$, then $x\in A,x\in C$, hence with $x\notin (A\cap B\cap C)$, one must have $x\notin B$. So, $x\in A\cap (C\setminus(B\cap C))$).

    Hence, express in probability, we have:
    \begin{align}
        \PP(A\cap (B\cup C))&=\PP((A\cap B)\sqcup ((A\cap C)\setminus(A\cap B\cap C))) = \PP(A\cap B)+\PP((A\cap C)\setminus(A\cap B\cap C))\\
        &=\PP(A)\PP(B)+\PP(A\cap C)-\PP(A\cap B\cap C) = \PP(A)\PP(B)+\PP(A)\PP(C)-\PP(A)\PP(B)\PP(C)\\
        &= \PP(A)(\PP(B)+\PP(C)-\PP(B)\PP(C)) = \PP(A)\PP(B\cup C)
    \end{align}
    Hence, we get that $A$ and $B\cup C$ are independent.

    \hfil

    On the other hand, if consider $(A\setminus B)\cap C$, we have the event being the same as $(A\cap C)\setminus(A\cap B\cap C)$ (if $x in (A\setminus B)\cap C$, we have $x \in A\cap C$, while $x \notin A\cap B\cap C$ since $x \notin B$, so $x in (A\cap C)\setminus (A\cap B\cap C)$; on the other hand, if $x\in (A\cap C)\setminus(A\cap B\cap C)$, since $x\in A$ and $x\in C$, $x\notin A\cap B\cap C$ implies $x\notin B$, hence $x \in (A\setminus B)\cap C$). So, its probability is given by:
    \begin{align}
        \PP((A\setminus B)\cap C)&=\PP((A\cap C)\setminus (A\cap B\cap C)) = \PP(A\cap C)-\PP(A\cap B\cap C)=\PP(A)\PP(C)-\PP(A)\PP(C)\PP(B)\\
        &= \PP(A)(1-\PP(B))\PP(C)
    \end{align}
    (Note: here we uses the fact that the events are independent).

    Now, notice that using similar logic, $A\setminus B=A\setminus (A\cap B)$, so $\PP(A\setminus B)=\PP(A\setminus (A\cap B))=\PP(A)-\PP(A\cap B)=\PP(A)-\PP(A)\PP(B)=\PP(A)(1-PP(B))$, then the above equality tells that $\PP((A\setminus B)\cap C)=\PP(A\setminus B)\PP(C)$, which $A\setminus B$ and $C$ are independent.
\end{proof}

\newpage
\section*{6}
\begin{ques}\label{q6}
    3.5.28:

    Let $\{A_1,A_2,...,A_n\}$ be an independent set of events and $\PP(A_i)=p_i$, $1\leq i\leq n$.
    \begin{itemize}
        \item[(a)] What is the probability that at least one of the events $A_1,A_2,...,A_n$ occurs?
        \item[(b)] What is the probability that none of the events $A_1,A_2,...,A_n$ occurs?
    \end{itemize}
\end{ques}

\begin{proof}

    \hfil

    \begin{itemize}
        \item[(a)] let $B_i$ be the event denoting ``smallest $1\leq j\leq n$ with event $A_j$ happens is $A_i$'' (i.e. $A_i$ happens, while all preceding events didn't), then notice that $B_1,...,B_n$ are all mutually disjoint (since if $i\neq j$, WLOG say $i<j$, then $B_i$ must have $A_i$ happens, while $B_j$ must have $A_i$ not happen).
        \item[(b)] 
    \end{itemize}
\end{proof}

\newpage
\section*{7}
\begin{ques}\label{q7}
\end{ques}

\begin{proof}
\end{proof}

\newpage
\section*{8}
\begin{ques}\label{q8}
\end{ques}

\begin{proof}
\end{proof}

\end{document}