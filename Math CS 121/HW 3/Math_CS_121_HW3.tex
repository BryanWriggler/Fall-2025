\documentclass{article}
\usepackage{../../Self_Style}

\title{Math CS 121 HW 3}
\author{Zih-Yu Hsieh}
\date{\today}

\begin{document}
\maketitle

\section*{1}
\begin{ques}\label{q1}
    3.3.18:

    Suppose that $10$ good and $3$ dead batteries are mixed up. Jack tests them one by one, at random and without replacement. But before testing the fifth battery he realizes that he does not remember whether the first one tested is good or is dead. All he remembers is that the last three that were tested were all good. What is the probability that the first one is also good?
\end{ques}

\begin{proof}
    Let $A$ denotes the event of the first battery being good, while $B$ denotes the event of the $2^\textmd{nd}$ to $4^\textmd{th}$ ones being good. Then, the scenario described has event $B$ being true, which serves as the condition. 

    Notice that $A \cap B$ denotes the event that the first four tested batteries are all good, which since there are total of $13$ batteries with $10$ being good (and $3$ being bad), $\PP(A\cap B) = \frac{\begin{pmatrix}10\\4\end{pmatrix}}{\begin{pmatrix}13\\4\end{pmatrix}}=\frac{42}{143}$ (since we need to choose $4$ good batteries out of the total of $10$ good ones, while there are toal of $13$ choose $4$ ways of doing so).

    Similarly, the probability of $B$ happening is given as follow (where we care about the second to the fourth ones):
    \begin{align}
        \PP(B)&=\PP(\textmd{First one being good}) \cdot \PP(\textmd{Next 2 to 4 being good})+\PP(\textmd{First one being bad})\cdot \PP(\textmd{Next 2 to 4 being good})\\
        &= \frac{10}{13}\cdot \frac{\begin{pmatrix}9\\3\end{pmatrix}}{\begin{pmatrix}12\\3\end{pmatrix} }+\frac{3}{13}\cdot\frac{\begin{pmatrix}10\\3\end{pmatrix} }{\begin{pmatrix}12\\3\end{pmatrix}} = \frac{60}{143}
    \end{align}
    (Note: After taking out the first battery without replacement, there are $12$ batteries left in each case. If the first one is good, only $9$ goods are left, while if the first one is bad, $10$ goods are left instead, that's why it's in the above form).

    Hence, the conditional probability $\PP(A|B)=\frac{\PP(A\cap B)}{\PP(B)}=\frac{42}{60}=\frac{7}{10}$.
\end{proof}

\newpage

\section*{2}
\begin{ques}\label{q2}
    3.3.19:

    A box contains $18$ tennis balls, of which eight are new. Suppose that three balls are selected randomly, played with, and after play are returned to the box. If other three balls are selected for play a second time, what is the probability that they are all new?
\end{ques}

\begin{proof}
    Given three balls are selected randomly initially, then there are four disjoint events: either all three are old (denoted as $A_0$), one is new and two are old (denoted as $A_1$), two are new and one is old (denoted as $A_2$), or all three are new (denoted as $A_3$). If we calculated their probability, we get (Note: there are initially $8$ new balls and $10$ old balls, total of $18$ balls. And here, we're choosing the balls without order):
    \begin{align}
        \PP(A_0)=\frac{\begin{pmatrix}10\\3\end{pmatrix}}{\begin{pmatrix}18\\3\end{pmatrix}}=\frac{5}{34}, \quad \PP(A_1)=\frac{\begin{pmatrix}10\\2\end{pmatrix}\begin{pmatrix}8\\1\end{pmatrix}}{\begin{pmatrix}18\\3\end{pmatrix}}=\frac{15}{34}, \quad \PP(A_2)=\frac{\begin{pmatrix}10\\1\end{pmatrix}\begin{pmatrix}8\\2\end{pmatrix}}{\begin{pmatrix}18\\3\end{pmatrix}}=\frac{35}{102}, \quad \PP(A_3)=\frac{\begin{pmatrix}8\\3\end{pmatrix}}{\begin{pmatrix}18\\3\end{pmatrix}}=\frac{7}{102}
    \end{align}
    Then, if $A_0$ happens, there are still $8$ new balls (since no new balls are selected); if $A_1$ happens, there are $7$ new balls left (since one new ball is selected and became an old one); if $A_2$ happens, there are $6$ new balls left (since two new balls are selected); similarly, if $A_3$ happens, there are only $5$ new balls left.

    So, let $W$ denotes the event that during the second selection (so, the selection has happened), all three balls newly selected are new, then it's probability is given as follow:
    \begin{align}
        \PP(W)&=\PP(W|A_0)\cdot \PP(A_0)+\PP(W|A_1)\cdot \PP(A_1)+\PP(W|A_2)\cdot \PP(A_2)+\PP(W|A_3)\cdot \PP(A_3)\\
        &= \frac{\begin{pmatrix}8\\3\end{pmatrix}}{\begin{pmatrix}18\\3\end{pmatrix}}\cdot \frac{5}{34}+\frac{\begin{pmatrix}7\\3\end{pmatrix}}{\begin{pmatrix}18\\3\end{pmatrix}}\cdot \frac{15}{34}+\frac{\begin{pmatrix}6\\3\end{pmatrix}}{\begin{pmatrix}18\\3\end{pmatrix}}\cdot\frac{35}{102}+\frac{\begin{pmatrix}5\\3\end{pmatrix}}{\begin{pmatrix}18\\3\end{pmatrix}}\cdot\frac{7}{102} = \frac{3185}{83232}\approx 0.0383
    \end{align}
    Here, $W$ is an intersection of ``all newly selected 3 balls are new'' and ``first selection of 3 balls had happened'', but since the second part is guaranteed to happen (or this becomes our new sample space), then the above is the desired probability (or there's no need to consider if the condition of $A_0\sqcup A_1\sqcup A_2\sqcup A_3$ had happened or not).

    (Note: The conditional probability is calculated based on the ``updated scenario'' and the corresponding number of new balls).
\end{proof}

\newpage
\section*{3}
\begin{ques}\label{q3}
    3.4.8:

    Urns I,II, and III contain three pennies and four dimes, two pennies and five dimes, three pennies and one dime, respectively. One coin is selected at random from each urn. If two of the three coins are dimes, what is the probability that the coin selected from urn I is a dime?
\end{ques}

\begin{proof}
    For this, since there are only one coin coming from each urn, while there are exactly two dimes and one penny eventually, hence it splits into three disjoint events: The penny is coming from I while the two dimes are from II and III (denoted as $A_I$), the penny is coming from II while the two dimes are from I and III (denoted as $A_{II}$), and the penny is coming from III while the two dimes are from I and III (denoted as $A_{III}$). As a result, they have the following probabilities(using the fact that we're choosing one from each urns, while I, II have 7 coins, and III has 4 coins):
    \begin{align}
        \PP(A_I)=\frac{3}{7}\cdot\frac{5}{7}\cdot\frac{1}{4}=\frac{15}{196},\quad \PP(A_{II})=\frac{4}{7}\cdot\frac{2}{7}\cdot\frac{1}{4}=\frac{8}{196},\quad \PP(A_{III})=\frac{4}{7}\cdot\frac{5}{7}\cdot\frac{3}{4}=\frac{60}{196}
    \end{align}
    Which, let $A$ denotes the event that eventually gets two dimes and one penny, we have $A=A_I\sqcup A_{II}\sqcup A_{III}$, hence $\PP(A)=\PP(A_I)+\PP(A_{II})+\PP(A_{III})=\frac{83}{196}$.

    \hfil

    Now, let $W$ denotes the event where the coin selected from urn I is a dime, this corresponds to event $A_{II},A_{III}$ if $A$ happens (while for $A_I$ urn I had gotten a penny, which is not possible to get a dime). Then, the probability of $W$ and $A$ happening together is given by:
    \begin{align}
        \PP(W \cap A)&=\PP(W|A_I)\cdot \PP(A_I)+\PP(W|A_{II})\cdot\PP(A_{II})+\PP(W|A_{III})\cdot\PP(A_{III})\\
        &= 0\cdot \frac{15}{196}+1\cdot\frac{8}{196}+1\cdot\frac{60}{196}=\frac{68}{196}
    \end{align}
    Hence, as a consequence the probability that $W$ happens given that $A$ happens, is $\PP(W|A)=\frac{\PP(W\cap A)}{\PP(A)} = \frac{68}{196}\cdot\frac{196}{83} = \frac{68}{83}$.
\end{proof}

\newpage

\section*{4}
\begin{ques}\label{q4}
    3.4.11:

    With probability of $1/6$ there are $i$ defective fuses aong 1000 fuses ($i=0,1,2,3,4,5$). If among $100$ fuses selected at random, none was defective, what is the probability of no defective fuses at all?
\end{ques}

\begin{proof}
    Let $A$ denotes the event that the $100$ selected fuses are all not defective, while let $B$ denotes the event where no fuses are defective. Then, to compute $\PP(B|A)$ here, using Bayes Rule we have $\PP(B|A)=\frac{\PP(A|B)\cdot\PP(B)}{\PP(A)}$.

    First, if given that $B$ happens (where there are no defective fuses), then $A$ would always happen (since the $100$ selected fuses will always be coming from all the non-defective fuses). So, $\PP(A|B)=1$.

    Then, the probability that $B$ happens (where there's no defective fuses, or $i=0$ for the notations given in the question) is given by $\PP(B)=\PP(\{i=0\}) = \frac{1}{6}$.

    Finally, the probability that $A$ happens (where the $100$ selected fuese are all non-defective), it can be classified into disjoint cases of having $i=0,1,...,5$ defective fuses. Since one of these events must happen (given that each $i$ is equally likely to happen with probability $1/6$, which the total probability of all cases of $i$ is $1$), by conditional probability, we get:
    \begin{align}
        \PP(A)=\sum_{j=0}^{5}\PP(A|\{i=j\})\cdot\PP(\{i=j\}) = \frac{1}{6}\sum_{j=0}^{5}\PP(A|\{i=j\})
    \end{align}
    Where, if there are $i=j$ number of defective fuses, then the probability of having all $100$ randomly selected fuses all being non-defective, is $\PP(A|\{i=j\})=\frac{\begin{pmatrix}1000-j\\100\end{pmatrix}}{\begin{pmatrix}1000\\100\end{pmatrix}}$. Hence, the above sum becomes:
    \begin{align}
        \PP(A)=\frac{1}{6}\sum_{j=0}^{5}\frac{\begin{pmatrix}1000-j\\100\end{pmatrix}}{\begin{pmatrix}1000\\100\end{pmatrix}} \approx 0.7807
    \end{align}
    Glue all the pieces together, we get:
    \begin{align}
        \PP(B|A)=\frac{\PP(A|B)\cdot\PP(B)}{\PP(A)} \approx \frac{1\cdot 1/6}{0.7807}\approx 0.2135
    \end{align}
\end{proof}

\newpage

\section*{5}
\begin{ques}\label{q5}
    3.5.21:

    Prove that if $A,B$, and $C$ are independent, then $A$ and $B \cup C$ are independent. Also show that $A\setminus B$ and $C$ are independent.
\end{ques}

\begin{proof}
    Suppose $A,B$, and $C$ are independent. Then we have $\PP(A\cap B)=\PP(A)\PP(B)$ (where the same logic applies to any two distinct pairs of events from $A,B,C$), and $\PP(A\cap B\cap C)=\PP(A)\PP(B)\PP(C)$.

    \hfil

    Now, if consider $A$ and $B\cup C$, since $B \cup C=B \sqcup (C\setminus B) = B\sqcup (C\setminus (B\cap C))$, then we have $\PP(B\cup C)=\PP(B)+\PP(C\setminus (B\cap C))=\PP(B)+\PP(C)-\PP(B\cap C)$. Also, since $A\cap (B\cup C) = (A\cap B)\sqcup (A\cap (C\setminus (B\cap C))) = (A\cap B)\sqcup ((A\cap C)\setminus (A\cap B\cap C))$ (since if $x \in A\cap (C\setminus(B\cap C))$, we have $x\in A,x\in C$, while $x\notin B$, so $x\in (A\cap C)\setminus(A\cap B\cap C)$; on the other hand, if $x\in (A\cap C)\setminus(A\cap B\cap C)$, then $x\in A,x\in C$, hence with $x\notin (A\cap B\cap C)$, one must have $x\notin B$. So, $x\in A\cap (C\setminus(B\cap C))$).

    Hence, express in probability, we have:
    \begin{align}
        \PP(A\cap (B\cup C))&=\PP((A\cap B)\sqcup ((A\cap C)\setminus(A\cap B\cap C))) = \PP(A\cap B)+\PP((A\cap C)\setminus(A\cap B\cap C))\\
        &=\PP(A)\PP(B)+\PP(A\cap C)-\PP(A\cap B\cap C) = \PP(A)\PP(B)+\PP(A)\PP(C)-\PP(A)\PP(B)\PP(C)\\
        &= \PP(A)(\PP(B)+\PP(C)-\PP(B)\PP(C)) = \PP(A)\PP(B\cup C)
    \end{align}
    Hence, we get that $A$ and $B\cup C$ are independent.

    \hfil

    On the other hand, if consider $(A\setminus B)\cap C$, we have the event being the same as $(A\cap C)\setminus(A\cap B\cap C)$ (if $x in (A\setminus B)\cap C$, we have $x \in A\cap C$, while $x \notin A\cap B\cap C$ since $x \notin B$, so $x in (A\cap C)\setminus (A\cap B\cap C)$; on the other hand, if $x\in (A\cap C)\setminus(A\cap B\cap C)$, since $x\in A$ and $x\in C$, $x\notin A\cap B\cap C$ implies $x\notin B$, hence $x \in (A\setminus B)\cap C$). So, its probability is given by:
    \begin{align}
        \PP((A\setminus B)\cap C)&=\PP((A\cap C)\setminus (A\cap B\cap C)) = \PP(A\cap C)-\PP(A\cap B\cap C)=\PP(A)\PP(C)-\PP(A)\PP(C)\PP(B)\\
        &= \PP(A)(1-\PP(B))\PP(C)
    \end{align}
    (Note: here we uses the fact that the events are independent).

    Now, notice that using similar logic, $A\setminus B=A\setminus (A\cap B)$, so $\PP(A\setminus B)=\PP(A\setminus (A\cap B))=\PP(A)-\PP(A\cap B)=\PP(A)-\PP(A)\PP(B)=\PP(A)(1-PP(B))$, then the above equality tells that $\PP((A\setminus B)\cap C)=\PP(A\setminus B)\PP(C)$, which $A\setminus B$ and $C$ are independent.
\end{proof}

\hfil
\section*{6}
\begin{ques}\label{q6}
    3.5.38:

    An urn contains $9$ red and $1$ blue balls. A second urn contains $1$ red and $5$ blue balls. One ball is removed from each urn at random and without replacement, and all of the remaining balls are put into a third urn. If we draw two balls randomly from the third urn, what is the probability that one of them is red and the other one is blue?
\end{ques}

\begin{proof}
    Here, let $R_1$ denotes removing red ball from urn 1, $R_1^c$ indicates an event of removing blue ball from urn 1, $R_2$ denotes removing red ball from urn 2, and $R_2^c$ denotes removing blue ball from urn $2$. Then, $\PP(R_1)=\frac{9}{10}$, $\PP(R_1^c)=1-\frac{9}{10}=\frac{1}{10}$, $\PP(R_2)=\frac{1}{6}$, and $\PP(R_2^c)=1-\frac{1}{6}=\frac{5}{6}$.

    Which, since the ball removing event is guaranteed to happen, then after the balls are removed, the scenario splits into four cases: $R_1\cap R_2$ (removing red ball from both urns), $R_1\cap R_2^c$ (removing red ball from urn 1, blue ball from urn 2), $R_1^c\cap R_2$ (removing blue ball from urn 1, red ball from urn 2), and $R_1^c\cap R_2^c$ (removing blue balls from both urns).

    Notice that the ball picking event for urn 1 and urn 2 are in fact independent (i.e. removing ball from one of the urns doesn't affect the removal of balls from the other urn), which implies that $\PP(R_1\cap R_2)=\PP(R_1)\PP(R_2)=\frac{3}{20}$, $\PP(R_1\cap R_2^c)=\PP(R_1)\PP(R_2^c)=\frac{3}{4}$, $\PP(R_1^c\cap R_2)=\PP(R_1^c)\PP(R_2)=\frac{1}{60}$, and $\PP(R_1^c\cap R_2^c)=\PP(R_1^c)\PP(R_2^c)=\frac{1}{12}$.

    \hfil

    Now, since after removing balls from both urns we mix them together to get a new pile, the remaining balls have the following data:
    \begin{itemize}
        \item If $R_1\cap R_2$ happens, there are total of $8$ red and $6$ blue balls left (removing 2 red balls in total).
        \item If $R_1\cap R_2^c$ or $R_1^\cap R_2$ happens, there are total of $9$ red and $5$ blue balls left (removing 1 red and 1 blue ball in total).
        \item Else if $R_1^c\cap R_2^c$ happens, there are total of $10$ red and $4$ blue balls left (removing 2 blue balls in total).
    \end{itemize}
    Let $T$ denotes the event of drawing 1 red and 1 blue ball out of the new pile, then its probability of happening (given the four disjoint cases of the first removal) is given as follow:
    \begin{itemize}
        \item $\PP(T|R_1\cap R_2)=\frac{\begin{pmatrix}8\\1\end{pmatrix}\begin{pmatrix}6\\1\end{pmatrix}}{\begin{pmatrix}14\\2\end{pmatrix}}=\frac{48}{91}$ (Choose 1 red out of 8 reds, 1 blue out of 6 blues, and total of $14C2$ ways of choosing two balls).
        \item $\PP(T|R_1\cap R_2^c)=\PP(T|R_1^c\cap R_2)=\frac{\begin{pmatrix}9\\1\end{pmatrix}\begin{pmatrix}5\\1\end{pmatrix}}{\begin{pmatrix}14\\2\end{pmatrix}}=\frac{45}{91}$ (Choose 1 red out of 9 reds, 1 blue out of 5 blues).
        \item $\PP(T|R_1^c\cap R_2^c)=\frac{\begin{pmatrix}10\\1\end{pmatrix}\begin{pmatrix}4\\1\end{pmatrix}}{\begin{pmatrix}14\\2\end{pmatrix}}=\frac{40}{91}$ (Choose 1 red out of 10 reds, 1 blue out of 4 blues).
    \end{itemize}
    Hence, the total probability that $T$ happens is:
    \begin{align}
        \PP(T)&=\PP(T|R_1\cap R_2)\PP(R_1\cap R_2)+\PP(T|R_1\cap R_2^c)\PP(R_1\cap R_2^c)+\PP(T|R_1^2\cap R_2)\PP(R_1^c\cap R_2)+\PP(T|R_1^c\cap R_2^c)\PP(R_1^c\cap R_2^c)\\
        &= \frac{3}{20}\cdot\frac{48}{91}+\left(\frac{3}{4}+\frac{1}{60}\right)\cdot\frac{45}{91}+\frac{1}{12}\cdot\frac{40}{91} = \frac{193}{390}\approx 0.4949
    \end{align}
\end{proof}

\newpage
\section*{7}
\begin{ques}\label{q7}
    3.Rev.5:

    Professor Stern has three cars. The probability that on a given day car $1$ is operative is $0.95$, that car $2$ is operative is $0.97$, and that car $3$ is operative is $0.85$. If Professor Stern's cars operate independently, find the probability that on next Thanksgiving day (a) all three of his cars are operative; (b) at least one of his cars is operative; (c) at most two of his cars are operative; (d) none of his cars is operative.
\end{ques}

\begin{proof}

    Let $A$ denotes car $1$ is operative, $B$ denotes car $2$ is operative, while $C$ denotes car $3$ is operative. So, $\PP(A)=0.95,\PP(B)=0.97$, and $\PP(C)=0.85$, while the events $A,B,C$ are independent.

    \begin{itemize}
        \item[(a)] The event that all three of the cras are operative is $A \cap B\cap C$. Which, by independence its probability is $\PP(A\cap B\cap C)=\PP(A)\PP(B)\PP(C)=\frac{31331}{40000}\approx 0.7833$.
        \item[(b)] At least one of the cars are operative is actually denoted as $A \cup B\cup C$ (since at least one of the car is operative, so one of $A,B,C$ must happen). Then, its probability is given as follow:
        \begin{align}
            \PP(A\cup B\cup C)&=\PP(A)+\PP(B)+\PP(C) - \PP(A\cap B)-\PP(A\cap C) -\PP(B\cap C)+\PP(A\cap B\cap C)\\
            &= \PP(A)+\PP(B)+\PP(C)-\PP(A)\PP(B)-\PP(A)\PP(C)-\PP(B)\PP(C) + \PP(A)\PP(B)\PP(C)\\
            &= \frac{39991}{40000}\approx 0.9998
        \end{align}
        \item[(c)] Let $D$ denotes the event that at most two of the cars are operative, $D$ has a complement being ``more than two cares are operative'', hence its complement is the event that all three cars are operative, which is given by part (a). So, since $D^c=A\cap B\cap C$, we have $\PP(D)=1-\P(D^c)=1-\PP(A\cap B\cap C)=\frac{8669}{40000}\approx 0.2167$.
        \item[(d)] If $E$ denotes the event that none of the cars are operative, since $E^c$ denotes the event that at least some cars are operative (more precisely, at least one of the car is operating), so $E^c=A\cup B\cup C$ (which has probability given in part (b)). Hence, $\PP(E)=1-\PP(E^c)=1-\PP(A\cup B\cup C)=\frac{9}{40000}\approx 0.0002$.
    \end{itemize}
\end{proof}

\hfil

\section*{8}
\begin{ques}\label{q8}
    3.Rev.19:

    A student at a certain university wilil pass the oral Ph.D. qualifying examination if at least two of the three examiners pass her or him. Past experience shows that 
    
    (a) $15\%$ of the studentst who take the qualifying exam are not prepared, and

    (b) each examiner will independently pass $85\%$ of the prepared and $20\%$ of the unprepared students.

    Kevin took his Ph.D. qualifying exam with Professors Smith, Brown, and Rose. What is the probability that Professor Rose has passed Kevin if we know that neither Professor Brown nor Professor Smith has passed him? Let $S,B$, and $R$ be the respective events that Professors Smith, Brown, and Rose have passed Kevin. Are these three events independent? Are they conditionally independent given that Kevin is prepared?

    \begin{defn}
        With three events $A,B,C$, the two events $B,C$ are conditionally independent given $A$, if $\PP(B\cap C | A)=\PP(B|A)\PP(C|A)$.
    \end{defn}
\end{ques}

\begin{proof}

    \hfil

    \textbf{1. Kevin passing with Professor Rose, given he's failed by Professor Brown and Smith:}

    There are two cases to consider: Either Kevin prepared for his quals (denoted as $A$), or Kevin didn't prepare for his quals (given by $A^c$). Where, $\PP(A^c)=0.15$ (since $15\%$ of the students are not prepared), while $\PP(A)=1-\PP(A^c)=0.85$.

    Then, given that the professors pass the students independently, given that Professors Broan and Smith had not passed Kevin, it doesn't affect the probability of Keving being passed by Professor Rose. Which, the probability is solely dependent on Kevin's preparation. So, it's given as follow:
    \begin{align}
        \PP(R|S\cap B)=\PP(R)=\PP(R|A)\PP(A) + \PP(R|A^c)\PP(A^c)
    \end{align}
    And, since the professors has probability of $85\%$ passing a prepared students, and $20\%$ passing an unprepared students, $\PP(R|A)=0.85$, while $\PP(R|A^c)=0.2$. So, the above becomes $\PP(R|S\cap B)=0.85\cdot 0.85+0.2\cdot 0.15 = 0.7525$.
    
    \hfil

    \textbf{2. Independence of the Professors passing Kevin:}

    Since by description the probability of each professor's decision on passing the dtudents are independent, then givn the passing status from the other profesors, the same professor still has the sam probability of passing. Hence, we get:
    \begin{align}
        \PP(R|S) = \PP(R)\implies \PP(R\cap S)=\PP(R|S)\PP(S)=\PP(R)\PP(S)\\
        \PP(B|S) = \PP(B)\implies \PP(B\cap S)=\PP(B|S)\PP(S)=\PP(B)\PP(S)\\
        \PP(R|B) = \PP(R)\implies \PP(R\cap B)=\PP(R|B)\PP(B)=\PP(R)\PP(B)
    \end{align}
    This first concludes the pairwise independence. Also, follow from the logic in the previous part, we deduce the following:
    \begin{align}
        \PP(R|S\cap B)=\PP(R)\implies \PP(R\cap (S\cap B))=\PP(R|S\cap B)\PP(S\cap B)= \PP(R)\PP(S)\PP(B)
    \end{align}
    This finishes the last proof of the three events $S,B,R$ being independent.

    \hfil

    \textbf{3. Independence (or not) given that Kevin is Prepared:}

    Here, given that Kevin is prepared (which is event $A$), each professor then would have the same probability of $85\%$ passing Kevin (since $85\%$ chance the prepared ones will get passed). So, $\PP(S|A)=\PP(R|A)=\PP(B|A)=0.85$.

    Now, notice that if considering $\PP(R\cap S|A) = \frac{\PP(R\cap S\cap A)}{\PP(A)}$, since $\PP(A)=0.85$, while $R$ and $S$ would happen independently with probability of $85\%$ after event $A$ happens (since professors have $85\%$ chance of passing a prepared students), showing that $(R\cap A)$ and $(S\cap A)$ are independent. Then we get that $\PP(R\cap S\cap A)=\PP((R\cap A)\cap (S\cap A))=\PP(R\cap A)\PP(S\cap A) = 0.85\cdot \PP(A)\cdot 0.85\cdot \PP(A) = 0.85^4$. THis shows that $\PP(R\cap S|A)=\frac{\PP(R\cap S\cap A)}{\PP(A)}=\frac{0.85^4}{0.85}=0.85^3$.

    Yet, if consider the fact that $\PP(R|A)=\PP(S|A)=0.85$, we have $\PP(R|A)\PP(S|A)=0.85^2$, showing that $\PP(R\cap S|A) = 0.85^3\neq 0.85^2=\PP(R|A)\PP(S|A)$, so the three events $R,S,B$ are not conditionally independent given $A$ happens (or given that Kevin has prepared).
\end{proof}

\end{document}