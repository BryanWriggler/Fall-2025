\documentclass{article}
\usepackage{../../Self_Style}

\title{Phys 20AL Week 3: Pendulum Improvement}
\author{Zih-Yu Hsieh}
\date{\today}

\begin{document}
\maketitle

\section{Aim for Experiment}
The experiment is to measure the amplitude dependence of the pendulum's period, experimenting with how the maximum initial angle affects the period of the pendulum. It also aims for testing the accuracy of the provided formula, where with maximum angle $\theta_0$, the period formula is given as follow:
$$T(\theta_0)=2\pi\sqrt{\frac{l}{g}}\left(1+\frac{1}{16}\theta_0^2+\frac{11}{3072}\theta_0^4+...\right)$$ 

\section{Experimental Setup}
\subsection{Experiment with Electronic Detector}
Equipments include: Stand, Clamp, string, spherical mass, protractor, meter stick , detector, computer.

The following is the steps for setup:
\begin{itemize}
    \item[1.] Setup the stand at hte edge of the table.
    \item[2.] Tie one end of the string onto the clamp, pass the other end through the hole at the top of the stand, and clip the clamp onto the stand (to fix the length of the string as pendulum's arm).
    \item[3.] Fix the center of the protractor at the top of the stand at where the pivot of the pendulum is at.
    \item[4.] Fix the spherical mass at the free end of the string (treated as mass of the simple pendulum).
    \item[5.] Fix the detector at the bottom of the stand (the location slightly lower than the spherical mass's position) for ease to detect the period of the spherical mass.
\end{itemize}


\subsection{Experimnt without Electronic Detector}
Equipments include: Stand, Clamp, string, spherical mass, protractor, meter stick , timer, phone (use as camera).

The setup is nearly identical with the previous one, except for not fixing a detector on the stand.

\end{document}