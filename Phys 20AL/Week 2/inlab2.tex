\documentclass{article}
\usepackage{../../Self_Style}

\title{Phys 20 Week 2: Friction}
\author{Zih-Yu Hsieh}
\date{\today}

\begin{document}
\maketitle

\section{Aim for Experiment}

Test out the coefficients of static and kinetic friction of the cart on ramps made of different materials.

\section{Experimental Setup}
Equipments include: Meter stick, tape, masses, mass scale, stand, wooden ramp, clamp, paper, plastic board, protractor.

\begin{itemize}
    \item[1.] Fix the stand next to the table.
    \item[2.] Fix the clip onto the wooden ramp, clamp the clip onto the stand, and let one side of the wooden ramp lay on the table. This is used for adjusting the angle of the ramp.
    \item[3.] Fix the protractor's center at the edge of the wooden ramp that's touching the table, which is used to measure the angle.
    \item[4.] Tape the (desired) surface for the experiment (ex: plastic board, paper, etc.) onto the wooden ramp.
\end{itemize}

\section{Experimental Procedures}

Given 3 different materials, test out their coefficients of static / kinetic friction.

For this, assume constant acceleration.

For static friction:
\begin{itemize}
    \item[1.] Measure the mass of the materials placed on the ramp.
    \item[2.] Put the materials on the ramp, and start increasing the angle.
    \item[3.] When the materials start to travel (i.e. no longer static), record the angle (called \emph{cricial angle}).
    \item[4.] For each material / mass, do the trial 5 times.
\end{itemize}

Either choose $3$ materials, or $3$ masses.

\hfil

For kinetic friction:
\begin{itemize}
    \item[1.] Measure the mass of the materials again.
    \item[2.] Put the materials on the ramp, and slowly increase the angle until it starts moving.
    \item[3.] When it starts moving, measure the time it takes to slide off a certain distance (while keeping the angle fixed).
    \item[4.] Measure the angle of the ramp after fixing the angle (when the materials are sliding). 
\end{itemize}


\newpage

\end{document}