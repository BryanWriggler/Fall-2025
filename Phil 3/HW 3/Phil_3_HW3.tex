\documentclass{article}
\usepackage{../../Self_Style}

\title{Phil 3 HW 3i}
\author{Zih-Yu Hsieh}
\date{\today}

\begin{document}
\maketitle

\begin{ques}\label{q1}
Apply Criterion 2 the following deductive argument, i.e. determine
whether or not it is valid. Briefly explain your answer.

All Arcturians are good kissers.

Some Arcturians have several mouths.

$\therefore$ Some things with several mouths are good kissers.
\end{ques}

\begin{proof}
    The argument is valid, since by the second premise some Arcturians have several mouths, and by first premise they're good kissers (since they're Arturians). So, these Arcturians with several mouths are good kissers, which is the conclusion.
\end{proof}

\hfil

\begin{ques}\label{q2}
Apply Criterion 2 the following deductive argument, i.e. determine
whether or not it is valid. Briefly explain your answer.

All Olympic sprinters are fast runners.

All fast runners have excellent cardiovascular fitness.

$\therefore$ All Olympic sprinters have excellent cardiovascular fitness.
\end{ques}

\begin{proof}
    This argument is valid, since all Olympic Sprinters are fast runners by premis 1, while by premise 2 they must have excellent cardiovascular fitness (since they're fast runners), hence all Olympic sprinters must have excellent cardiovascular fitness.
\end{proof}

\hfil

\begin{ques}\label{q3}
Apply Criterion 2 the following deductive argument, i.e. determine
whether or not it is valid. Briefly explain your answer.

Some houses have a kitchen.

Some houses have a bathroom.

$\therefore$ Some houses have both a kitchen and a bathroom.
\end{ques}

\begin{proof}
    This is not valid, since one can always build a house that only contain one of kitchen or bathroom, but not both to provide a counterexample.
\end{proof}

\newpage

\begin{ques}\label{q4}
Apply Criterion 2 to the following deductive argument, i.e. determine whether or not it is valid. Briefly explain your answer.

All philosophers are clear and careful thinkers.

Adams is a clear and careful thinker.

$\therefore$ Adams is a philosopher.
\end{ques}

\begin{proof}
    This is not valid, since Adams can be not a philosopher to still be a clear and careful thinker (Ex: Lawyer, Engineer, Scientists, Mathematicians, etc. in their own field).
\end{proof}

\hfil

\begin{ques}\label{q5}
Consider the following deductive argument. Is it valid? Why or
why not?

No Americans like hamburgers.

Joe is not American.

$\therefore$ Joe likes hamburgers.
\end{ques}

\begin{proof}
    This is not valid, since there exists non-American people who don't like hamburgers (from my experience, my Grandpa).
\end{proof}

\hfil

\begin{ques}\label{q6}
Consider the following deductive argument. Is it valid? Why or
why not?

No Americans like hamburgers.

Joe likes hamburgers.

$\therefore$ Joe is not American.
\end{ques}

\begin{proof}
    This is valid, since the first statement guarantees that ``Being an American'' implies ``the person doesn't like hamburgers', then with the second statement guarantess Joe likes hamburgers. If Joe is an American, then Jow must also dont like hamburgers, which is a contradiction. So, Joe cannot be an American.
\end{proof}

\newpage

\begin{ques}\label{q7}
Consider the following deductive argument.

If Topeka is in the United States, then it is either in the continental United
States, in Alaska, in Hawaii, or in a US territory.

Topeka is not in Alaska.

Topeka is not in Hawaii.

Topeka is not in a US territory.

Topeka is in the United States.

$\therefore$ Topeka is in the continental United States.

Answer the following questions and briefly explain each answer. a) Are the
premises true? b) Given the truth of the premises, is the conclusion guaranteed
to be true too? c) How relevant are the premises to the conclusion?
\end{ques}

\begin{proof}

    \hfil

    \begin{itemize}
        \item[a)] The premises are all true, since US contains only the conteinental US, Alaska, Hawaii, and US territory. While Topeka (as the capital of Kansas, which is in the US) is not in Alaska, Hawaii, nor any other US territory.
        \item[b)] Yes, the conclusion is guaranteed to be true, since the $5^\textmd{th}$ statement guarantees Topeka to be in the US, hence by $1^\textmd{st}$ statement it's either in the conteinental US, Alaska, Hawaii, or the US territory. Bu, statement $2,3,4$ guarantees that Topeka is not in Alaska, Hawaii, nor in any US territory. Hence, the only option left is Topeka being in the conteinental US, which is our conclusion.
        \item[c)] These premises are all related to the conclusions (since they're all necessary to derive the conclusion, when using the logical derivation in part b)).
    \end{itemize}
\end{proof}

\hfil

\begin{ques}\label{q8}
Evaluate the following inductive argument with respect to Criterion 2. Is it a strong inductive argument? Why or why not?

We saw an eagle in the park.

There are only two species of eagles in the park, bald and golden.

Golden eagles are commonly seen in the park.

Bald eagles are rarely seen in the park.

$\therefore$ The eagle we saw in the park was a golden eagle.
\end{ques}

\begin{proof}
    Yes, it's a strong inductive argument, since statement 2 guarantees that there are only two species of eagles (either bald or golden), while statements 3 and 4 ensures that there is a high probability meeting golden eagles, while a low probability meeting bald eagles. Hence, there is a high probability seeing an eagle, which concluding ``we saw a golden eagle in the park'' is having a high probability of being right.
\end{proof}

\newpage

\begin{ques}\label{q9}
Consider the following statement: Thomas Barrett is the president
of the United States.

a) Give a deductive argument for this statement that satisfies Criterion 2
(i.e. it is deductively valid).

b) Give a deductive argument for this statement that fails to satisfies Criterion 2 (i.e. it is not deductively valid).

c) Can you give an argument for this statement that is sound? Briefly explain
your answer.

Note: If you’d like, you can use the same argument more than once on this
exercise.
\end{ques}

\begin{proof}

    \hfil

    \begin{itemize}
        \item[a)] Thomas Barrett is a philosopher.
        
        All philosopher is a president of the US.

        $\therefore$ Thomas Barrett is the president of the US.
        \item[b)] Thomas Barrett is a philosopher.
        
        None of the philosopher is a president of the US.

        $\therefore$ Thomas Barrett is the president of the US.
        \item[c)]  No, it's not possible to give a sound argument for this, since it requires all premises to be true, while it's deductively valid, showing that the conclusion must also be true. But, since the conclusion is false, this argument being sound would cause a contradiction, so any argument involving this statement as conclusion must not be sound.
        
        On the other hand, this statement is false so any argument invovling this as a premise is automatically not sound either.
    \end{itemize}
\end{proof}

\newpage

\begin{ques}\label{q10}
Consider these sentence letters. P : Pam is going. Q: Quincy
is going. R: Richard is going. S: Sara is going. Express each of the following
sentences in the language of propositional logic. (Make sure that your answers
are well-formed formulas.)

a) Either Pam is not going or Quincy is not going.

b) Richard is going if Pam and Quincy are both not going.

c) If Richard is going, then if Pam is not going, Quincy is going.

d) Pam is not going if Quincy is.

e) Richard and Quincy are going if and only if either Pam or Sara is going.

f) Neither Richard nor Sara is going.

g) If both Quincy and Richard are going, then Pam is going.

h) Either Quincy is going, or both Pam and Sara are going.

i) Either Pam and Sara are going or Quincy is going but Richard isn’t.

j) Although Richard and Quincy are going, Pam and Sara are not.
\end{ques}

\begin{proof}

    \hfil

    \begin{itemize}
        \item[a)] $(\sim P \vee \sim Q)$ (where $\vee$ is ``or'').
        \item[b)] $((\sim P \wedge \sim Q)\implies R)$. (where $\wedge$ is ``and'').
        \item[c)] $(R \implies (\sim P \implies Q))$.
        \item[d)] $(Q \implies \sim P)$.
        \item[e)] $((R \wedge Q)\iff (P \vee S))$.
        \item[f)] $(\sim R\wedge \sim S)$.
        \item[g)] $((Q \wedge R)\implies P)$.
        \item[h)] $(Q \vee (P \wedge S))$.
        \item[i)] $((P \wedge S) \vee (Q \wedge \sim R))$.
        \item[j)] $((R \wedge Q) \wedge (\sim P \wedge \sim S))$.
    \end{itemize}
\end{proof}

\end{document}
