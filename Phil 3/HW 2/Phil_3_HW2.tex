\documentclass{article}
\usepackage{../../Self_Style}

\title{Phil 3 HW 2}
\author{Zih-Yu Hsieh}
\date{\today}

\begin{document}
\maketitle

\begin{ques}\label{q1}
\textbf{Exercise 1.} Put the following arguments in standard form. (Hint: All of these are complex arguments!)

\begin{enumerate}
\item[(a)] The square of any number n is divisible by n. Hence the square of any even number is even, since by the principle just mentioned it must be divisible by an even number, and any number divisible by an even number is even.

\item[(b)] Neither the butler nor the maid did it. That leaves the chauffeur or the cook. But the chauffeur was at the airport when the murder took place, so the cook is the only one without an alibi. It’s logical to conclude that the cook did it.

\item[(c)] You don’t need to worry about cold weather in Santa Barbara. There hasn’t ever been a severe winter storm here, and so there probably never will be.

\item[(d)] Since he agreed to take care of the dog while they were away, that’s what he should do. So if he neglects the dog, then he’s clearly at fault.

\item[(e)] He couldn’t have saved the file, since his computer crashed before he had a chance. Since he couldn’t have saved it, there’s no point in blaming him. We should therefore cut him some slack.

\item[(f)] There are many extraterrestrial civilizations in our galaxy that are technologically advanced. Hence we have the ability to detect signals generated by extraterrestrials, since by the fact just mentioned there are extraterrestrials who can generate electromagnetic signals powerful enough to be detected on earth, and we have the tools to detect such signals.
\end{enumerate}
\end{ques}
\begin{proof}

    \hfil

    \begin{itemize}
        \item[(a)] The square of any number $n$ is divisible by $n$.
        
        $\therefore$ If $n$ is an even number, square of $n$ is divisible by an even number.
        
        Any number divisble by an even number is even.
        
        $\therefore$ Square of any even number is even.

        \item[(b)] Neither the butler nor the maid did the murder.
        
        $\therefore$ The chauffeur or the cook did it.

        The chauffeur was at the airport when the murder took place.

        $\therefore$ The cook is the only oone without an alibi.

        $\therefore$ The cook did it.

        \item[(c)] There hasn't ever been a severe winter storm in Santa Barbara.
        
        $\therefore$ There probably never will be.

        $\therefore$ You don't need to worry about cold weather in Santa Barbara.

        \item[(d)] He agreed to take care of the dog while they were away.
        
        $\therefore$ That's what he should do.

        $\therefore$ If he neglects the dog, then he's clearly at fault.

        \item[(e)] His computer crashed before he had a chance.
        
        $\therefore$ He couldn't have saved the file.

        $\therefore$ There's no point in blaming him.

        $\therefore$ We should cut him some slack.

        \item[(f)] There are many extraterrestrial civilizations in our galaxy that are technologically advanced.
        
        $\therefore$ There are extraterrestrials who can generate electromaagnetic signals powerful enough to be detected on Earth.

        We have the tools to detect such signals.

        $\therefore$ We have the ability to detect signals generated by extraterrestrials.
    \end{itemize}
\end{proof}

\pagebreak

\begin{ques}\label{q2}
\textbf{Exercise 2.} Put the following arguments in standard form. (Hint: These arguments have an implicit conclusion and/or premises!)

\begin{enumerate}
\item[(a)] If there had been a highway patrol officer back there, they would have shown up on my radar detector.

\item[(b)] I can’t study unless my roommates turn down the music; and they just won’t turn it down.

\item[(c)] Jacob never went to college, which just goes to show that you don’t need a college degree to be successful.

\item[(d)] If he were really my friend, he wouldn’t ignore my texts.
\end{enumerate}
\end{ques}

\begin{proof}
    
    \hfil

    \begin{itemize}
        \item[(a)] If there had been a highway patrol officer back there, they would have shown up on my radar detector.
        
        (Implicit) No highway patrol officer showed up on my radar detector.
        
        (Implicit) $\therefore$ There's no highway patrol officer back there.

        \item[(b)] I can't study unless my roommates turn down the music.
        
        They just won't turn it (the music) down.

        $\therefore$ I can't study.

        \item[(c)] Jacob never went to college.
        
        (Implicit) Jacob is successful.

        $\therefore$ You don't need a college degree to be successful.

        \item[(d)] If he were really my friend, he wouldn't ignore my texts.
        
        (Implicit) He ignored my texts.

        (Implicit) $\therefore$ He wasn't really my friend.
    \end{itemize}
\end{proof}

\pagebreak

\begin{ques}\label{q3}
\textbf{Exercise 3.} Give an argument in normal English prose with:

\begin{enumerate}
\item[(a)] no premise indicator words
\item[(b)] no conclusion indicator words
\item[(c)] no premise indicator words and no conclusion indicator words
\end{enumerate}

Make sure that they really are arguments! Then put your arguments from (a) and (b) and (c) into standard form.
\end{ques}

\begin{proof}

    Statements:

    \begin{itemize}
        \item[(a)] I can't study in UCen. UCen is too noisy and I can't study with loud noise.
        \item[(b)] Notice that my bank account is empty. I can't pay my college tuition.
        \item[(c)] I didn't study for my hard midterm. I fail the midterm.
    \end{itemize}

    \hfil

    Standard forms:

    \begin{itemize}
        \item[(a)] UCen is too noisy.
        
        I can't study with loud noise.

        $\therefore$ I can't study in UCen.

        \item[(b)] My Bank account is empty.
        
        $\therefore$ I can't pay my college tuition.

        \item[(c)] I didn't study for my hard midterm.
        
        $\therefore$ I fail the midterm.
    \end{itemize}
\end{proof}

\pagebreak

\begin{ques}\label{q4}
\textbf{Exercise 4.} Find an argument in real life. The argument can be from a conversation that you have with someone, something you read somewhere, something that you hear on TV, etc. The only rule here is that it must be an argument!

\begin{enumerate}
\item[(a)] First describe the argument as you found it in real life. (Ex: Samantha said “We need to go to Trader Joe’s. There’s nothing in the fridge and you’re getting hangry.”)
\item[(b)] Then put the argument into standard form.
\end{enumerate}
\end{ques}

\begin{proof}
    \begin{itemize}
        \item[(a)] My friend once said his classes have large workload, and he already had too much to work on outside of classes, so he should drop the class.
        \item[(b)] His classes have large workload.
        
        He already had too much to work on outside of classes.

        (Implicitly) He should decrease his workload.

        $\therefore$ He should drop the class.
    \end{itemize}
\end{proof}

\hfil

\begin{ques}\label{q5}
\textbf{Exercise 5.} Consider the following statement: Thomas Barrett is the president of the United States.

\begin{enumerate}
\item[(a)] Give an argument for this statement (i.e. an argument with this statement as its conclusion) that satisfies Criterion 1 (i.e. it has true premises).
\item[(b)] Give an argument for this statement that does not satisfy Criterion 1.
\end{enumerate}
\end{ques}

\begin{proof}
    \begin{itemize}
        \item[(a)] Thomas Barrett is a US citizen.
        
        A US citizen is eligible to be the president.

        $\therefore$ Thomas Barrett is the prisident of the US. (Which the implication is of course false).

        \item[(b)] All philosopher is a president of the US. (this premise is false)
        
        Thomas Barrett is a philosopher.

        $\therefore$ Thomas Barrett is the president of the US.
    \end{itemize}
\end{proof}

\end{document}