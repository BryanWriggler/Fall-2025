\documentclass{article}
\usepackage{../../Self_Style}

\title{Phil 3 HW 1}
\date{\today}

\begin{document}
\maketitle

\begin{ques}\label{q1}
    Consider the following statements:
    \begin{itemize}
        \item[(a)] UCSB is the best university in California.
        \item[(b)] UCSB is not the best university in California. 
    \end{itemize}
\end{ques}

\begin{proof}
    
    \hfil

    \begin{itemize}
        \item[(a)]
        
        Weather at UCSB is stable.

        Having stable weather is one of the most important factors for a university in California to be the best.

        $\therefore$ UCSB is the best university in California.

        \item[(b)]
        
        UCSB doesn't have a lot of fundings compared to Stanford.

        Stanford is in California.

        Having more fundings implies the university is better.

        $\therefore$ UCSB is not the best university in California.
    \end{itemize}
\end{proof}

\hfil

\begin{ques}
    For each of your arguments in a) and b) from \textbf{Exercise 1}, say whether or not you think your argument is a good argument or a bad argument. Do you think that your argument successfully proves the conclusion? Why do you think this?
\end{ques}\label{q2}

\begin{proof}

    \hfil

    \begin{itemize}
        \item[(a)] I believe my argument does form a good argument. Assuming that UCSB has stable weather, and stable weather is an important factor for a good university, then we can conclude that UCSB is a good university in California. Strengthen the statement by saying it's the indicating factor of the ``best'' university in California, then one can conclude that UCSB is the "best" university in California.
        \item[(b)] Again, I believe my argument does form a good argument. Assuming UCSB doesn't have enough fundings compared to Stanford, and how funding creates an order for the university's quality, while Stanford is in fact in California, this indicates that Stanford is a better Californian university than UCSB, hence UCSB is not the best university in California.
    \end{itemize}

    (In my opinion, to form a good argument it doesn't ``care'' about the truthfulness of the assumptions / premises, it cares more about the logical process of how conclusion is derived from the assumptions).
\end{proof}

\hfil

\begin{ques}
    Put the following simple arguments in standard form.
    \begin{itemize}
        \item[(a)] We need more morephine. We've got 32 casualties and there are only 12 doses of morphine left.
        \item[(b)] The earth is approximately 93 million miles from the sun. The moon is about 250,000 miles from the earth. So the moon is about 250,000 miles closer to the sun than the earth is.
        \item[(c)] Since my keys aren't in my pocket, I must have left them at home, because I did drive home last night and I couldn't have driven home without my keys.
        \item[(d)] Samantha will do well in Phil 3, since she works hard and hard work is all it takes to do well in Phil 3.
        \item[(e)] The sidewalk is wet, so it must have rained. If it hadn't rained, the sidewalk would be dry.
        \item[(f)] That dog must be friendly. He's wagging his tail and licking everyone's hands.
        \item[(g)] Climate models predict that hurricanes will become more frequent if global temperatures keep rising. Since global temperatures have been rising, we should expect more hurricanes.
        \item[(h)] A new species of tree surviving on that island seems unlikely, since the climate is extremely dry, and since no new plant species have been recorded there in decades.       
    \end{itemize}
\end{ques}\label{q3}

\begin{proof}

    \hfil
    
    \begin{itemize}
        \item[(a)] We've got 32 casualties.
        
        There are only 12 doses of morphine left.

        $\therefore$ We need more morephine.
        \item[(b)] The earth is approximately 93 million miles from the sun.
        
        The moon is about 250,000 miles from the earth.

        $\therefore$ the moon is about 250,000 miles closer to the sun than the earth is.

        (Unfortunately not true though, since difference in scalars is different from difference in vectors in higher dimensional space).
        \item[(c)] My keys aren't in my pocket.
        
        I did drive home last night.

        I couldn't have driven home without my keys.

        $\therefore$ I must have the keys at home.
        \item[(d)] Samantha works hard.
        
        Har work is all it takes to do well in Phil 3.

        $\therefore$ Samantha will do well in Phil 3.
        \item[(e)] The sidewalk is wet.
        
        If it hadn't rained, the sidewalk would be dry.

        $\therefore$ it must have rained.
        \item[(f)] That dog is wagging his tail and licking everyone's hands.
        
        $\therefore$ That dog must be friendly.
        \item[(g)] Climate models predict that hurricanes will become more frequent, if global temperatures keep rising.
        
        Global temperatures have been rising.

        $\therefore$ we should expect more hurricanes.
        \item[(h)] The climate on that island is extremely dry.
        
        No new plant species have been recorded there in decades.

        $\therefore$ A new species of tree surviving on that island seems unlikely.
    \end{itemize}
\end{proof}

\end{document}