\documentclass{article}
\usepackage{../../Self_Style}

\title{Phys 115A HW 2}
\author{Zih-Yu Hsieh}
\date{\today}

\begin{document}
\maketitle

\section*{1}
\begin{ques}\label{q1}
To build a little more experience with separation of variables, consider the heat equation
in one spatial dimension:
\[
\frac{1}{k}\frac{\partial u(x, t)}{\partial t} = \frac{\partial^2 u(x, t)}{\partial x^2}
\]
where $u(x, t)$ corresponds to the temperature of an object with thermal conductivity $k$.

1. Using separation of variables, $u(x, t) = u_x(x)u_t(t)$, and separation constant $-\alpha^2$, separate the heat equation into ordinary differential equations for $u_t$ and $u_x$.

2. Find the general solution to the differential equation satisfied by $u_x(x)$.

3. Find the general solution to the differential equation satisfied by $u_t(t)$.

4. Write down the general solution $u(x, t)$ to the heat equation.
\end{ques}

\begin{proof}

    \hfil

    \begin{enumerate}
        \item Given the separation of variables $u(x,t)=u_x(x)u_t(t)$, plug into the formula we get the following:
        \begin{align}
            \frac{1}{k}\frac{\partial}{\partial t}(u_x(x)u_t(t))=\frac{\partial^2}{\partial x^2}(u_x(x)u_t(t))&\implies \frac{1}{k}\frac{d u_t}{dt}u_x(x)=\frac{d^2u_x}{dx^2}u_t(t)\\
            &\implies \frac{1}{k}\cdot \frac{1}{u_t(t)}\frac{du_t}{dt}=\frac{1}{u_x(x)}\frac{d^2u_x}{dx^2}
        \end{align}
        Notice that the line aboe have the left side being dependent only on $t$, while the right side is solely dependent on $x$, hence the whole equation must be constant. With the separation constant given by $-\alpha^2$, we get the following two ordinary differential equations:
        \begin{align}
            &\frac{1}{k}\cdot \frac{1}{u_t(t)}\frac{du_t}{dt}=\frac{1}{u_x(x)}\frac{d^2u_x}{dx^2}=-\alpha^2\\
            &\implies \begin{cases}
                \frac{du_t}{dt}=-k\alpha^2 u_t(t)\\
                \frac{d^2u_x}{dx^2}=-\alpha^2u_x(x)
            \end{cases}
        \end{align}

        \hfil

        \item For $u_x(x)$, since it satisfies $\frac{d^2u_x}{dx^2}+\alpha^2u_x(x)=0$ (or can rewrite as $\frac{d^2 u_x}{dx^2}-(i\alpha)^2 u_x(x)=0$), then the general solution of this is given by $Ae^{i\alpha x}+Be^{-i\alpha x}$, where $A,B\in\CC$ are arbitrary constants (depending on the initial condition).
        
        (Note: If want real solution instead, it depends on the constant $\alpha$, so here we'll provide the most general form since $\alpha$ is not specified).

        \hfil
        
        \item For $u_t(t)$, since it satisfies $\frac{du_t}{dt}=-k\alpha^2 u_t(t)$, then the general solution  $u_t(t)=Ke^{-k\alpha^2 t}$, where $K\in\CC$ is an arbitrary constant (depending on the initial condition).
        
        (Note: If want real solution just take the arbitrary constants in $\RR$ instead).

        \hfil
        
        \item If plug back all the information in part 2 and 3, we get that $u(x,t) = u_x(x)u_t(t)=A'e^{-k\alpha^2t + i\alpha x}+B'e^{-k\alpha^2t - i\alpha x}$, for a separable solution with separation constant $-\alpha^2$.
        
        Which in general, any arbitrary (at most countable, and converging) linear combination of these solutions (each with distinct $\alpha$) still serve as a solution to the given heat equation. But, since the use of Fourier series could describe most of the common solutions, then suppose the domain of the function $u(x,t)$ is limited to $[-\pi,\pi]$ (and given specific boundary condition, like the function at $t=0$), one can specifically choose $\alpha = n\in\ZZ$ as coefficients, and yield the following expression by Fourier Series:
        \begin{align}
            u(x,t)=\sum_{n=-\infty}^{\infty}c_n Ce^{-kn^2t+inx},\quad c_n :=\int_{-\pi}^{\pi}u(x,0)e^{-inx}dx
        \end{align}
    \end{enumerate}
\end{proof}

\hfil

\section*{2}
\begin{ques}\label{q2}
Prove the following three theorems:

1. For normalizable solutions, the separation constant $E$ must be real. Hint: Write $E$
(in Griffiths eq. 2.7, i.e. $\Psi(x, t) = \psi(x)e^{-iEt/\hbar}$) as $E_0 + i\Gamma$, with $E_0$ and $\Gamma$ real, and
show that $\Gamma$ must be zero if Griffiths eq. 1.20 (i.e.
\[
\int_{-\infty}^{\infty} |\Psi(x, t)|^2 dx = 1
\]
is to hold for all $t$.

2. The time-independent wavefunction $\psi(x)$ can always be taken to be real (unlike
$\Psi(x, t)$, which is necessarily complex). This doesn’t mean that every solution to the
time-independent Schrödinger equation is real; what it says is that if you’ve got one
that is not, it can always be expressed as a linear combination of solutions (with the
same energy) that are. So you might as well stick to $\psi$’s that are real. Hint: if $\psi(x)$
satisfies Griffiths eq. 2.5 (i.e.
\[
-\frac{\hbar^2}{2m}\frac{d^2\psi}{dx^2} + V\psi = E\psi
\]
for a given $E$, so too does its complex conjugate, and hence also the real linear combinations
$\psi+\psi^*$ and $i(\psi-\psi^*)$.

3. If $V(x)$ is an even function (that is, $V(-x) = V(x)$) then $\psi(x)$ can also be taken to
be either even or odd. Hint: If $\psi(x)$ satisfies Griffiths eq. 2.5, for a given $E$, then so
too does $\psi(-x)$, and hence also the even and odd linear combinations $\psi(x)\pm\psi(-x)$.
\end{ques}

\begin{proof}

    \hfil

    \begin{enumerate}
        \item Suppose $\Psi(x,t)$ is a separable solution to the Schrödinger equation that is a scaled version of a wave function (so it's normalizable), then since $\Psi(x,t)=\psi(x)e^{-iE t/\hbar}$ for some constant $E\in\CC$, then $E=E_0+i\Gamma$ for some $E_0,\Gamma\in\RR$. Now, suppose the solution $\Psi$ itself is normalized, we have the following:
        \begin{align}
            1&=\int_{-\infty}^{\infty}|\Psi(x,t)|^2 dx=\int_{-\infty}^{\infty}\left(\psi(x)e^{-i(E_0+i\Gamma)t/\hbar}\right)\left(\psi^*(x)\left(e^{-i(E_0+i\Gamma)t/\hbar}\right)^*\right)dx\\
            &= \int_{-\infty}^{\infty}|\psi(x)|^2\cdot e^{2\Gamma t/\hbar}dx = e^{2\Gamma t/\hbar}\int_{-\infty}^{\infty}|\psi(x)|^2 dx
        \end{align}
        Since $\int_{-\infty}^{\infty}|\psi(x)|^2 dx$ must stay as constant for all $t\in\RR$, for the whole expression to be $1$ for all $t\in\RR$, one must have $e^{2\Gamma t/\hbar}$ being a constant also. Hence we must have $\Gamma=0$, showing that $E=E_0+i\Gamma=E_0\in\RR$.

        \hfil

        \item Suppose $\psi(x)$ satisfies the time independent Schrödinger equation, we have the following:
        \begin{align}
            -\frac{\hbar^2}{2m}\frac{d^2\psi}{dx^2}+V\psi=E\psi
        \end{align}
        Where both $V,E\in\RR$. Hence, if taken the complex conjugation, we get:
        \begin{align}
            \left(-\frac{\hbar^2}{2m}\frac{d^2\psi}{dx^2}\right)^*+(V\psi)^*=(E\psi)^*\implies -\frac{\hbar^2}{2m}\frac{d^2\psi^*}{dx^2}+V\psi^*=E\psi^*
        \end{align}
        Hence, $\psi^*$ is also a solution to the Time-independent Schrödinger equation. As a result, since $\psi(x)=\Real(\psi)(x)+i \Imag(\psi)(x)$, where $\Real(\psi)(x)=\frac{1}{2}(\psi(x)+\psi^*(x))$, and $\Imag(\psi)(x)=\frac{1}{2i}(\psi(x)-\psi^*(x))$, then since linear combinations of solutions to a differential equation is still a solution (and the fact that $\psi:\RR\rightarrow \CC\cong \RR^2$ is differentiable, one must have both its real and imaginary parts being differentiable also), then $\Real(\psi)(x),\Imag(\psi)(x)$ are both solutions to the Time-Independent Schrödinger equation. Hence, with these two components being real, $\psi(x)$ can be written as linear combinations of real solutions.

        \hfil

        \item Given that $V(x)$ is even, then $V(x)=V(-x)$. Now, if we consider $\psi(-x)$ instead, then we have $\frac{d}{dx}\psi(-x)=-\psi'(-x)$, and $\frac{d^2}{dx^2}\psi(-x)=\frac{d}{dx}\left(-\psi'(-x)\right) = \psi''(-x)$. Hence, since $\psi(x)$ is a solution to the Time-Independent Schrödinger equation by assumption, we have $-\frac{\hbar^2}{2m}\psi''(x)+V(x)\psi(x)=E\psi(x)$, then plug in $u=-x$ instead, we get $-\frac{\hbar^2}{2m}\psi''(-x)+V(-x)\psi(-x)=E\psi(-x)$, which implies the following:
        \begin{align}
            -\frac{\hbar^2}{2m}\frac{d^2}{dx^2}(\psi(-x))+V(x)\psi(-x)=E\psi(-x)
        \end{align}
        Hence, $\psi(-x)$ is also a solution. 

        Then, as a result $\frac{\psi(x)+\psi(-x)}{2}$ and $\frac{\psi(x)-\psi(-x)}{2}$ are both solutions to the time-independent Schrödinger equation, where the first one is even, the latter one is odd, and $\psi(x)=\frac{\psi(x)+\psi(-x)}{2}+\frac{\psi(x)-\psi(-x)}{2}$. Hence, any solution $\psi(x)$ can be expressed as linear combinations of odd and even functions.
    \end{enumerate}
\end{proof}

\newpage

\section*{3  (ND)}
\begin{ques}\label{q3}
Show that $E$ must exceed the minimum value of $V(x)$, for every normalizable solution to
the time-independent Schrödinger equation. What is the classical analog to this statement?

Hint: Rewrite Griffiths eq. 2.5 (i.e.
\[
-\frac{\hbar^2}{2m}\frac{d^2\psi}{dx^2} + V\psi = E\psi
\]
in the form
\[
\frac{d^2\psi}{dx^2} = \frac{2m}{\hbar^2}[V(x) - E]\psi;
\]
if $E < V_{\min}$, then $\psi$ and its second derivative always have the same sign – argue that such
a function cannot be normalized.
\end{ques}

\begin{proof}
    Suppose the contrary that $E< V_{\min}$, then rewrite the Time-Independent Schrödinger equation, we get:
    \begin{align}
        \frac{d^2\psi}{dx^2} = \frac{2m}{\hbar^2}(V(x) - E)\psi
    \end{align}
    Where, since $E<V_{\min}$, then let $\epsilon=V_{min}-E>0$, there are several cases to consider (which we can assume both $\psi,\psi',\psi''$ and $V$ are all continuous):
    \begin{itemize}
        \item[1)] If at some point $x_0\in\RR$, one have $\psi(x_0)>0$ and $\psi'(x_0)\geq 0$, then we have $\psi''(x_0) =\frac{2m}{\hbar^2}(V(x_0)-E)\psi(x_0)\geq \frac{2m}{\hbar^2}(V_{\min}-E)\psi(x_0)=\frac{2m\epsilon}{\hbar^2}\psi(x_0)>0$. Then, by continuity of $\psi''$, there exists some radius $r>0$, such that within $(x_0-r,x_0+r)$ we have $\psi''(x)>0$ (and by decreasing the radius and taking the closed interval instead, WLOG can assume $\psi''(x)>0$ on some closed interval $[x_0-r,x_0+r]$). Then, we get that $\psi'(x_0+\delta) = \int_{x_0}^{x_0+\delta}\psi''(x)dx + \psi'(x_0) > \psi'(x_0)\geq 0$ for all $0\leq \delta\leq r$, showing that $\psi(x_0+\delta)=\int_{x_0}^{x_0+\delta}\psi'(x)dx+\psi(x_0)>\psi(x_0)>0$ for $0\leq \delta\leq r$, therefore $\psi(x)$ is increasing on the closed interval $[x_0,x_0+r]$.
        
        Now, plug in $x_0+r$ instead of $x_0$ into the system, we again reach the same argument, hence one can inductively push the interval to be larger (say $\psi(x)$ is still increasing on $[x_0+nr, x_0+(n+1)r]$ for any $n \in\NN$), and show that $\psi(x)$ is always increasing for $x\geq x_0$. However, with $\psi(x_0)>0$, this shows that $\lim_{x\rightarrow\infty}\psi(x)$ is not converging to $0$, which violates the normalization condition (since in case for $\int_{-\infty}^{\infty}|\psi|^2dx=1$, $\psi(x)$ must converge to $0$ as $x\rightarrow\infty$).

        \item[2)] If at some point $x_0\in\RR$, one have $\psi(x_0),\psi'(x_0)<0$ instead, then we have $\psi''(x_0)=\frac{2m}{\hbar}(V(x)-E)\psi(x_0)\leq \frac{2m}{\hbar}(V_{\min}-E)\psi(x_0)=\frac{2m\epsilon}{\hbar}\psi(x_0)<0$. Then, by continuity of $|psi''$, again there exists some radius $r>0$, such that within $[x_0-r,x_0+r]$ we have $\psi''(x)<0$. So, $\psi'(x_0+\delta)=\int_{x_0}^{x_0+\delta}\psi''(x)dx+\psi'(x_0)<\psi'(x_0)<0$ for all $0\leq \delta\leq r$, showing that $\psi(x_0+\delta)=\int_{x_0}^{x_0+\delta}\psi'(x)dx+\psi(x_0)<\psi(x_0)<0$ for $0\leq \delta\leq r$, therefore $\psi(x)$ is decreasing on the closed interval $[x_0,x_0+r]$.
        
        Again using induction one can push the interval indefinitely, showing that $\psi(x)$ is decreasing for $x\geq x_0$, but since $\psi(x_0)<0$, this shows that $\lim_{x\rightarrow\infty}\psi(x)$ is again not converging to $0$, which violates the normalization condition.

        \item[3)] If the reference point $x_0\in\RR$ satisfies either $\psi(x_0)>0,\psi'(x_0)<0$ (or the other case $\psi(x_0)<0$ and $\psi'(x_0)\geq 0$), we claim that it reduces to the previous two cases: For definiteness we'll demonstrate the case $\psi(x_0)>0$ and $\psi'(x_0)<0$ (since the other case is done by reversing the inequality). 
        
        In this case $\psi''(x_0)>0$ (since $\psi'',\psi$ have the same sign based on the differential equation), hence we derive that $\psi'(x)$ is increasing within some small interval containing $x_0$ (but maintaining negative), while in this region $\psi(x)$ is constantly decreasing.

        If after some position $x_0+r$ (when $\psi(x_0+r)> 0$) we have $\psi'(x_0+r)\geq 0$, then it reduces to case 1).
        
        Else (the other case), it implies whenever $\psi(x_0+r)>0$ we have $\psi'(x_0+r)<0$, showing that $\psi$ is constantly decrasing on some interval. If $\psi(x_0+\delta)>0$ for all $x_0+\delta\in[x_0,x_0+r]$, then $\psi$ must crossover to be negative at some point $x_0+r$ (or $\psi(x_0+r)<0$), which reduces to case 2).

        (\textbf{Rmk:} Notice that it's not possible to have $\psi(x)=\psi'(x)=0$, since with $V(x)$ being continuous, the given Schrödinger equation in fact has a unique solution; then, with $\psi(x)=0$ trivially satisfy the given equation and this specific initial condition, we have $\psi(x)=0$, yet this contradicts the normalization condition.)
    \end{itemize}
    Hence, in either case one must have $\lim_{x\rightarrow\infty}\psi(x)\neq 0$ (either diverges or converges to some other nonzero real numbers), but this violates the normalization condition, so it reaches a contradiction.

    Hence, we must have $E\geq V_{\min}$.

    \hfil
    
    Finally,  as a classical analogy to it, it's with the object (or the mass) must have its energy being at least the minimum of the potential energy in the position (since the kinetic energy $K$ in classical mechanics cannot be lower than $0$, with total energy $E=K+V(x)$, then $K=E-V\geq 0$, showing that $E\geq V(x)$ at the given position, hence within the mass's movable range, $E\geq V(x)$, showing $E\geq V_{\min}$).
\end{proof}

\hfil

\section*{4}
\begin{ques}\label{q4}
You prepare a particle in the infinite square well (with walls at $x = 0, a$) in an initial state
described by a linear combination of two stationary states,
\[
\Psi(x, 0) = A[\psi_2(x) + \psi_3(x)]
\]
where $\psi_2, \psi_3$ are given by Griffiths eq. 2.31 (i.e. $\psi_n(x) = \sqrt{\frac{2}{a}}\sin\left(\frac{n\pi x}{a}\right)$) with $n = 2, 3$
respectively.

(a) Normalize $\Psi(x, 0)$, i.e., find $A$.

(b) Find $\Psi(x, t)$ and $|\Psi(x, t)|^2$. Express the latter as a sinusoidal function of time in
terms of the variable $\omega = \pi^2\hbar / 2ma^2$.

(c) Compute $\langle x \rangle$. Is this an interesting function of time? Note: the integral is painful,
but please work your way through it by hand and show your work. Integration by
parts and trigonometric identities (including one that Griffiths uses while proving
orthogonality of the stationary states) are both your friends!

(d) Compute $\langle p \rangle$.

(e) If you measured the energy of the particle, what values might you get, and what is
the probability of getting them? What is the expectation value of $H$, and how does
this relate to the energies $E_2$ and $E_3$ of $\psi_2$ and $\psi_3$?
\end{ques}

\begin{proof}

    \hfil

    \begin{itemize}
        \item[(a)]Given $\Psi(x,0)=A\sqrt{\frac{2}{a}}\left(\sin(\frac{2\pi x}{a})+\sin(\frac{3\pi x}{a})\right)$ on the well's domain $[0,a]$ (while $0$ everywhere else). For it to satisfy normalization condition, we aim to find $A$ so that $\int_{-\infty}^{\infty}|\Psi(x,0)|^2 dx=1$. Which, we get the below formula (since outside of $[0,a]$ it's $0$ constantly, we can ignore them):
        \begin{align}
            &1=\frac{2|A|^2}{a}\int_{0}^{a}\left(\sin^2\left(\frac{2\pi x}{a}\right)+\sin^2\left(\frac{3\pi x}{a}\right)+2\sin\left(\frac{2\pi x}{a}\right)\sin\left(\frac{3\pi x}{a}\right)\right)dx\\
            &\implies \frac{a}{2|A|^2}= \int_{0}^{a}\frac{1-\cos(\frac{4\pi x}{a})}{2}dx + \int_{0}^{a}\frac{1-\cos(\frac{6\pi x}{a})}{2}dx + \int_{0}^{a}\left(\cos\left(\frac{\pi x}{a}\right)-\cos\left(\frac{5\pi x}{a}\right)\right)dx\\
            &\implies \frac{a}{2|A|^2} = \frac{a}{2}+\frac{a}{2}+0 = a \implies |A|^2=\frac{1}{2}\implies |A|=\frac{1}{\sqrt{2}}
        \end{align}

        \hfil

        \item[(b)] Recall that for the time dependence, each separable solution $\Psi_n(x,t)=\sqrt{\frac{2}{a}}\sin(\frac{n\pi x}{a})e^{-iE_n t/\hbar}$, where $E_n=\frac{1}{2m}\left(\frac{\hbar n\pi}{a}\right)^2$. Hence, the full general form of $\Psi(x,t)=A(\psi_2(x)+\psi_3(x))$ is given as follow (given that $w:=\frac{\pi^2\hbar}{2ma^2}$):
        \begin{align}
            \Psi(x,t)&=\frac{1}{\sqrt{2}}\cdot\sqrt{\frac{2}{a}}\left(\sin\left(\frac{2\pi x}{a}\right)\exp\left(-4i\frac{\pi^2 \hbar}{2ma^2}t\right)+\sin\left(\frac{3\pi x}{a}\right)\exp\left(-9i\frac{\pi^2 \hbar}{2ma^2}t\right)\right)\\
            &= \frac{1}{\sqrt{a}}\left(\sin\left(\frac{2\pi x}{a}\right)e^{-4iwt}+\sin\left(\frac{3\pi x}{a}\right)e^{-9iwt}\right)
        \end{align}
        Which, $|\Psi(x,t)|^2 = \Psi(x,t)\cdot \Psi^*(x,t)$ is given as follow:
        \begin{align}
            |\Psi(x,t)|^2 &= \frac{1}{a}\left(\sin\left(\frac{2\pi x}{a}\right)e^{-4iwt}+\sin\left(\frac{3\pi x}{a}\right)e^{-9iwt}\right)\left(\sin\left(\frac{2\pi x}{a}\right)e^{-4iwt}+\sin\left(\frac{3\pi x}{a}\right)e^{-9iwt}\right)^*\\
            &= \frac{1}{a}\left(\sin^2\left(\frac{2\pi x}{a}\right)+\sin^2\left(\frac{3\pi x}{a}\right)+\sin\left(\frac{2\pi x}{a}\right)\sin\left(\frac{3\pi x}{a}\right)\left(e^{5iwt}+e^{-5iwt}\right)\right)\\
            &= \frac{1}{a}\left(\sin^2\left(\frac{2\pi x}{a}\right)+\sin^2\left(\frac{3\pi x}{a}\right)+2\sin\left(\frac{2\pi x}{a}\right)\sin\left(\frac{3\pi x}{a}\right)\cos(5wt)\right)
        \end{align}
        As a remark, both $\Psi(x,t)$ and $|\Psi(x,t)|^2$ are $0$ when $x\notin [0,a]$ based on the property of infinite square well.

        \hfil

        \item[(c)] Here to compute $\left<x\right>$, since outside of the well's domain $[0,a]$ we have $|\Psi(x,t)|^2=0$ (since $\Psi(x,t)=0$ whene $x\notin [0,a]$ due to the infinite square well property). Then, the expectation value is given as:
        \begin{align}
            \left<x\right> = \int_{0}^{a}x|\Psi(x,t)|^2dx= \int_{0}^{a}\frac{x}{a}\left(\sin^2\left(\frac{2\pi x}{a}\right)+\sin^2\left(\frac{3\pi x}{a}\right)+2\sin\left(\frac{2\pi x}{a}\right)\sin\left(\frac{3\pi x}{a}\right)\cos(5wt)\right)dx
        \end{align}
        We'll compute each integral individually. Each result as the following:
        \begin{align}
            A&:=\int_{0}^{a}x\sin^2\left(\frac{2\pi x}{a}\right)dx = \int_{0}^{a}x\cdot \frac{1-\cos(\frac{4\pi x}{a})}{2}dx = \frac{1}{4}x^2\bigg|_{0}^{a}-\frac{1}{2}\int_{0}^{a}x\cos\left(\frac{4\pi x}{a}\right)dx\\
            &= \frac{a^2}{4} - \frac{a}{8\pi}x\sin\left(\frac{4\pi x}{a}\right)\bigg|_{0}^{a}+\frac{a}{8\pi}\int_{0}^{a}\sin\left(\frac{4\pi x}{a}\right)dx = \frac{a^2}{4}
        \end{align}

        \begin{align}
            B&:=\int_{0}^{a}x\sin^2\left(\frac{3\pi x}{a}\right)dx = \int_{0}^{a}x\cdot\frac{1-\cos(\frac{6\pi x}{a})}{2}dx = \frac{1}{4}x^2\bigg|_{0}^{a}-\frac{1}{2}\int_{0}^{a}x\cos\left(\frac{6\pi x}{a}\right)dx\\
            &= \frac{a^2}{4}-\frac{a}{12\pi}x\sin\left(\frac{6\pi x}{a}\right)\bigg|_{0}^{a}+\frac{a}{12\pi}\int_{0}^{a}\sin\left(\frac{4\pi x}{a}\right)dx = \frac{a^2}{4}
        \end{align}

        \begin{align}
            C&:= \int_{0}^{a}x\cdot 2\sin\left(\frac{2\pi x}{a}\right)\sin\left(\frac{3\pi x}{a}\right)dx=\int_{0}^{a}x\left(\cos\left(\frac{\pi x}{a}\right)-\cos\left(\frac{5\pi x}{a}\right)\right)dx\\
            &= \left(\frac{a}{\pi}x\sin\left(\frac{\pi x}{a}\right)\bigg|_{0}^{a}-\int_{0}^{a}\frac{a}{\pi}\sin\left(\frac{\pi x}{a}\right)dx\right)-\left(\frac{a}{5\pi}x\sin\left(\frac{5\pi x}{a}\right)\bigg|_{0}^{a}-\int_{0}^{a}\frac{a}{5\pi}\sin\left(\frac{5\pi x}{a}\right)dx\right)\\
            &= \frac{a^2}{\pi^2}\cos\left(\frac{\pi x}{a}\right)\bigg|_{0}^{a}-\frac{a^2}{25\pi^2}\cos\left(\frac{5\pi x}{a}\right)\bigg|_{0}^{a} = \frac{2a^2}{25\pi^2}-\frac{2a^2}{\pi^2} = \frac{-48a^2}{25\pi^2}
        \end{align}
        (Note: In $A,B$ it uses the fact that $\sin^2(x)=\frac{1-\cos(2x)}{2}$, integration by parts, and integrating $\sin,\cos$ over their period / multiple of their periods are $0$; While in part $C$ it uses the trigonometric identity that $2\sin(a)\sin(b)=\cos(a-b)-\cos(a+b)$, and integration by parts).

        Which, with these pices, we get the following for expectation value:
        \begin{align}
            \left<x\right>= \frac{1}{a}\left(A+B+C\cos(5wt)\right) = \frac{a}{2}-\frac{48a}{25\pi^2}\cos(5wt)
        \end{align}
        Which, the expected position of the particle is in fact oscillating.

        \hfil

        \item[(d)] To compute $\left<p\right>=\int_{-\infty}^{\infty}\Psi^*(x,t) \cdot \frac{\hbar}{i}\frac{\partial}{\partial x}(\Psi(x,t))dx$, we get the following equation (Note: since $\Psi^*(x,t)$ is also only possibly nonzero on $x\in [0,a]$, we'll ignore other parts of the integration):
        \begin{align}
            \left<p\right>&=\frac{\hbar}{i}\int_{0}^{a}\frac{1}{\sqrt{a}}\left(\sin\left(\frac{2\pi x}{a}\right)e^{-4iwt}+\sin\left(\frac{3\pi x}{a}\right)e^{-9iwt}\right)^*\cdot \frac{\partial}{\partial x}\left(\frac{1}{\sqrt{a}}\left(\sin\left(\frac{2\pi x}{a}\right)e^{-4iwt}+\sin\left(\frac{3\pi x}{a}\right)e^{-9iwt}\right) \right) dx\\
            &= \frac{\hbar}{ia}\int_{0}^{a}\left(\sin\left(\frac{2\pi x}{a}\right)e^{4iwt}+\sin\left(\frac{3\pi x}{a}\right)e^{9iwt}\right)\left(\frac{2\pi}{a}\cos\left(\frac{2\pi x}{a}\right)e^{-4iwt}+\frac{3\pi}{a}\cos\left(\frac{3\pi x}{a}\right)e^{-9iwt}\right)dx\\
            &=\frac{\hbar}{ia}(\frac{2\pi}{a}\int_{0}^{a}\sin\left(\frac{2\pi x}{a}\right)\cos\left(\frac{2\pi x}{a}\right)dx + \frac{3\pi}{a}\int_{0}^{a}\sin\left(\frac{3\pi x}{a}\right)\cos\left(\frac{3\pi x}{a}\right)dx\\ 
            &+\frac{2\pi}{a}\int_{0}^{a}\sin\left(\frac{3\pi x}{a}\right)\cos\left(\frac{2\pi x}{a}\right)e^{5iwt}dx+\frac{3\pi}{a}\int_{0}^{a}\sin\left(\frac{2\pi x}{a}\right)\cos\left(\frac{3\pi x}{a}\right)e^{-5iwt}dx\\
            &= \frac{\hbar}{ia}(\frac{\pi}{a}\int_{0}^{a}\sin\left(\frac{4\pi x}{a}\right)dx + \frac{3\pi}{2a}\int_{0}^{a}\sin\left(\frac{6\pi x}{a}\right)dx +\frac{\pi}{a}e^{5iwt}\int_{0}^{a}\left(\sin\left(\frac{5\pi x}{a}\right)+\sin\left(\frac{\pi x}{a}\right)\right)dx\\
            & + \frac{3\pi}{2a}e^{-5iwt}\int_{0}^{a}\left(\sin\left(\frac{5\pi x}{a}\right)-\sin\left(\frac{\pi x}{a}\right)\right)dx)\\
            &= \frac{\hbar}{ia}\left(0+0+\frac{\pi}{a}e^{5iwt}\left(\frac{2a}{5\pi}+\frac{2a}{\pi}\right) + \frac{3\pi}{2a}e^{-5iwt}\left(\frac{2a}{5\pi}-\frac{2a}{\pi}\right)\right)\\
            &= \frac{\hbar}{ia}\left(\frac{\pi}{a}\cdot\frac{12a}{5\pi}e^{5iwt}-\frac{3\pi}{2a}\cdot\frac{8a}{5\pi}e^{-5iwt}\right) = \frac{\hbar}{ia}\cdot \frac{12}{5}\cdot 2i \sin(5wt) = \frac{24\hbar}{5a}\sin(5wt)
        \end{align}

        \hfil

        \item[(e)] Since $\Psi(x,0) = A(\psi_2(x)+\psi_3(x))$ with $A=\frac{1}{\sqrt{2}}$ in part (a), then in general $\Psi(x,t)=A(\psi_2(x,t)+\psi_3(x,t))$, where the two are the eigenfunctions corresponding to the Hamiltonian Operator $H$ with eigenvalue being $E_2$ and $E_3$.
        
        Recall that an energy of the infinite square well is discrete and must be an energy level corresponding to one of the eigenfunctions (i.e. being their eigenvalue), then the possible measurement of energy must be one of the $E_n$, $n\in\NN$. And, the general decomposition $\Psi(x,t)=\sum_{n=1}^{\infty}c_n \psi_n(x,t)$ satisfies $\PP(E_n)=|c_n|^2$, or the coefficient's norm square is the corresponding probability of measuring that energy state. 

        Hence, given $\Psi(x,t)= \frac{1}{\sqrt{2}}\psi_2(x,t)+\frac{1}{\sqrt{2}}\psi_3(x,t)$, since any $n\neq 2,3$ have coefficient $0$, there are $0$ probability of measuring them (more specifically, since they are not part of the wavefunction, then it's not expected to measure them). So, the only possible energy that can get measured are $E_2$ and $E_3$, and each with probability $\frac{1}{2}$ (since both have coefficient $c_2=c_3=\frac{1}{\sqrt{2}}$, so probability $|c_2|^2=|c_3|^2=\frac{1}{2}$).

        Now, to consider the expectation value of energy (or operator $H$), we get the following general form (recall that each $\psi_n(x,t)$ is an eigenfunction of $H$ with eigenvalue $E_n$, so $\hat{H}(\psi_k(x,t))=E_k \psi_k(x,t)$):
        \begin{align}
            \left<H\right> &= \int_{-\infty}^{\infty}\Psi^*(x,t)\cdot \hat{H}(\Psi(x,t))dx=\int_{-\infty}^{\infty}\left(\sum_{n=1}^{\infty}c_n \psi_n(x,t)\right)^*\cdot\hat{H}\left(\sum_{k=1}^{\infty}c_k \psi_k(x,t)\right)dx\\
            &= \int_{-\infty}^{\infty}\left(\sum_{n=1}^{\infty}c_n^* \psi_n^*(x,t)\right)\left(\sum_{k=1}^{\infty}c_k E_k \cdot \psi_k(x,t)\right)dx = \sum_{(n,k)\in\NN^2}c_n^* c_k E_k\int_{-\infty}^{\infty}\psi_n^*(x,t)\psi_k(x,t)dt\\
            &= \sum_{n=1}^{\infty}|c_n|^2E_n 
        \end{align}
        Where te last part uses the fact that the $\psi_n(x,t)$ form an orthonormal list, hence their integral inner product is $0$ unless $n,k$ matches (while in that case by normalization condition the integral yields $1$).

        Hence, for this specific $\Psi(x,t)=\frac{1}{\sqrt{2}}\psi_2(x,t)+\frac{1}{\sqrt{2}}\psi_3(x,t)$, we have $\left<H\right>=\frac{1}{2}\cdot E_2+\frac{1}{2}\cdot E_3 = \frac{4\hbar^2\pi^2}{4ma^2}+\frac{9\hbar^2\pi^2}{4ma^2} = \frac{13\hbar^2\pi^2}{4ma^2}$.
    \end{itemize}
\end{proof}

\hfil

\section*{5}
\begin{ques}\label{q5}
We have emphasized that overall phases of the wave function are irrelevant, since they
cancel out of physical quantities. But the relative phase matters! Imagine we change the
phase in the previous problem so that our initial state is
\[
\Psi(x, 0) = A[\psi_2(x) + e^{i\phi}\psi_3(x)]
\]
for some real constant $\phi$.

(a) Find $\Psi(x, t)$ and $|\Psi(x, t)|^2$.

(b) Find $\langle x \rangle$.

(c) Find $\langle p \rangle$.

(d) Discuss how these results differ from the case of $\phi = 0$. Consider especially the cases
$\phi = \pi/2$ and $\phi = \pi$.
\end{ques}

\begin{proof}
    First, given $\Psi(x,0)=A\sqrt{\frac{2}{a}}(\sin(\frac{2\pi x}{a})+e^{i\phi}\sin(\frac{3\pi x}{a}))$, to normalize this expression, we aim for $\int_{-\infty}^{\infty}|\Psi(x,t)|^2dx =1$ (and notice that outside of $[0,a]$, $\Psi(x,0)=0$, hence only need to consider a limited range). Which, we get:
    \begin{align}
        1&= \int_{0}^{a}\Psi^*(x,0)\Psi(x,0)dx=\frac{2|A|^2}{a}\int_{0}^{a}\left(\sin\left(\frac{2\pi x}{a}\right)+e^{i\phi}\sin\left(\frac{3\pi x}{a}\right)\right)^*\left(\sin\left(\frac{2\pi x}{a}\right)+e^{i\phi}\sin\left(\frac{3\pi x}{a}\right)\right)dx\\
        &= \frac{2|A|^2}{a}\left(\int_{0}^{a}\left(\sin^2\left(\frac{2\pi x}{a}\right)+\sin^2\left(\frac{3\pi x}{a}\right)\right)dx+2\cos(\phi)\int_{0}^{a}\sin\left(\frac{2\pi x}{a}\right)\sin\left(\frac{3\pi x}{a}\right)dx\right)\\
        &= \frac{2|A|^2}{a}\cdot a = 2|A|^2
    \end{align}
    As a result, $2|A|^2=1$, so $|A|=\frac{1}{\sqrt{2}}$. (Note: Here we're using some calculation in \textbf{Question \ref{q4}} for simplicity).

    \begin{itemize}
        \item[(a)] Given $\Psi(x,0)$ with constant $A=\frac{1}{\sqrt{2}}$, and each $\psi_n(x,t)$ has time dependence of a factor $e^{-iE_nt/\hbar}$, with $E_n = \frac{n^2\hbar^2\pi^2}{2ma^2}$, hence it has timde dependence of factor $e^{-in^2\frac{\hbar\pi^2}{2ma^2}t}$. Let $w:=\frac{\hbar\pi^2}{2ma^2}$ be the ``frequency'' for simplicity, we get each state $\psi_n(x,t)=\sqrt{\frac{2}{a}}\sin(\frac{n\pi x}{a})e^{-i\cdot n^2 wt}$. So, $\Psi(x,t)=A(\psi_2(x,t)+e^{i\phi}\psi_3(x,t))$ is as follow (for values within $[0,a]$; outside of this domain it's $0$ by the property of infinite square well):
        \begin{align}
            \Psi(x,t)=\frac{1}{\sqrt{a}}\left(\sin\left(\frac{2\pi x}{a}\right)e^{-i\cdot 4wt}+\sin\left(\frac{3\pi x}{a}\right)e^{-i\cdot (9wt-\phi)}\right)
        \end{align}
        Which, the corresponding $|\Psi(x,t)|^2=\Psi^*\Psi$ is given as follow (with the same limitation of infinite square welll from above):
        \begin{align}
            |\Psi(x,t)|^2&=\frac{1}{a}\left(\sin\left(\frac{2\pi x}{a}\right)e^{-i\cdot 4wt}+\sin\left(\frac{3\pi x}{a}\right)e^{-i\cdot (9wt-\phi)}\right)^*\left(\sin\left(\frac{2\pi x}{a}\right)e^{-i\cdot 4wt}+\sin\left(\frac{3\pi x}{a}\right)e^{-i\cdot (9wt-\phi)}\right)\\
            &= \frac{1}{a}\left(\sin\left(\frac{2\pi x}{a}\right)e^{i\cdot 4wt}+\sin\left(\frac{3\pi x}{a}\right)e^{i\cdot (9wt-\phi)}\right)\left(\sin\left(\frac{2\pi x}{a}\right)e^{-i\cdot 4wt}+\sin\left(\frac{3\pi x}{a}\right)e^{-i\cdot (9wt-\phi)}\right)\\
            &= \frac{1}{a}\left(\sin^2\left(\frac{2\pi x}{a}\right) + \sin^2\left(\frac{3\pi x}{a}\right) + \sin\left(\frac{2\pi x}{a}\right)\sin\left(\frac{3\pi x}{a}\right)\left(e^{i(5wt-\phi)}+e^{-i(5wt-\phi)}\right)\right)\\
            &= \frac{1}{a}\left(\sin^2\left(\frac{2\pi x}{a}\right) + \sin^2\left(\frac{3\pi x}{a}\right) + 2\sin\left(\frac{2\pi x}{a}\right)\sin\left(\frac{3\pi x}{a}\right)\cos(5wt-\phi)\right)
        \end{align}

        \hfil

        \item[(b)] To calculate $\left<x\right>$, consider the following integral (again, we'll consider only the range $[0,a]$ due to infinite square well's restriction):
        \begin{align}
            \left<x\right>&=\int_{0}^{a}x|\Psi(x,t)|^2dx\\
            &= \frac{1}{a}\left(\int_{0}^{a}x\sin^2\left(\frac{2\pi x}{a}\right)dx+\int_{0}^{a}x\sin^2\left(\frac{3\pi x}{a}\right)dx + \cos(5wt-\phi)\int_{0}^{a}x\sin\left(\frac{2\pi x}{a}\right)\sin\left(\frac{3\pi x}{a}\right)dx\right)\\
            &= \frac{1}{a}\left(\frac{a^2}{4}+\frac{a^2}{4}-\frac{48a^2}{25\pi^2}\cos(5wt-\phi)\right) = \frac{a}{2}-\frac{48a}{25\pi^2}\cos(5wt-\phi)
        \end{align}
        (Note: Here uses the calculation in \textbf{Question \ref{q4} part (c)}).

        
        \item[(c)] To calculate $\left<p\right>$, consider the following integral:
        \begin{align}
            \left<p\right>&=\frac{\hbar}{ia}\int_{0}^{a}\left(\sin\left(\frac{2\pi x}{a}\right)e^{-4iwt}+\sin\left(\frac{3\pi x}{a}\right)e^{-i(9wt-\phi)}\right)^*\frac{\partial}{\partial x}\left(\left(\sin\left(\frac{2\pi x}{a}\right)e^{-4iwt}+\sin\left(\frac{3\pi x}{a}\right)e^{-i(9wt-\phi)}\right) \right) dx\\ %1
            &= \frac{\hbar}{ia}\int_{0}^{a}\left(\sin\left(\frac{2\pi x}{a}\right)e^{4iwt}+\sin\left(\frac{3\pi x}{a}\right)e^{i(9wt-\phi)}\right)\left(\frac{2\pi}{a}\cos\left(\frac{2\pi x}{a}\right)e^{-4iwt}+\frac{3\pi}{a}\cos\left(\frac{3\pi x}{a}\right)e^{-i(9wt-\phi)}\right)dx\\ %2
            &=\frac{\hbar}{ia}(\frac{2\pi}{a}\int_{0}^{a}\sin\left(\frac{2\pi x}{a}\right)\cos\left(\frac{2\pi x}{a}\right)dx + \frac{3\pi}{a}\int_{0}^{a}\sin\left(\frac{3\pi x}{a}\right)\cos\left(\frac{3\pi x}{a}\right)dx\\ %3 
            &+\frac{2\pi}{a}\int_{0}^{a}\sin\left(\frac{3\pi x}{a}\right)\cos\left(\frac{2\pi x}{a}\right)e^{i(5wt-\phi)}dx+\frac{3\pi}{a}\int_{0}^{a}\sin\left(\frac{2\pi x}{a}\right)\cos\left(\frac{3\pi x}{a}\right)e^{-i(5wt-\phi)}dx\\ %4
            &= \frac{\hbar}{ia}(\frac{\pi}{a}\int_{0}^{a}\sin\left(\frac{4\pi x}{a}\right)dx + \frac{3\pi}{2a}\int_{0}^{a}\sin\left(\frac{6\pi x}{a}\right)dx +\frac{\pi}{a}e^{i(5wt-\phi)}\int_{0}^{a}\left(\sin\left(\frac{5\pi x}{a}\right)+\sin\left(\frac{\pi x}{a}\right)\right)dx\\ %5
            & + \frac{3\pi}{2a}e^{-i(5wt-\phi)}\int_{0}^{a}\left(\sin\left(\frac{5\pi x}{a}\right)-\sin\left(\frac{\pi x}{a}\right)\right)dx)\\ %6
            &= \frac{\hbar}{ia}\left(0+0+\frac{\pi}{a}e^{i(5wt-\phi)}\left(\frac{2a}{5\pi}+\frac{2a}{\pi}\right) + \frac{3\pi}{2a}e^{-i(5wt-\phi)}\left(\frac{2a}{5\pi}-\frac{2a}{\pi}\right)\right)\\%7
            &= \frac{\hbar}{ia}\left(\frac{\pi}{a}\cdot\frac{12a}{5\pi}e^{i(5wt-\phi)}-\frac{3\pi}{2a}\cdot\frac{8a}{5\pi}e^{-i(5wt-\phi)}\right) = \frac{\hbar}{ia}\cdot \frac{12}{5}\cdot 2i \sin(5wt-\phi) = \frac{24\hbar}{5a}\sin(5wt-\phi)
        \end{align}

        \hfil

        \item[(d)] In comparison to the answer in \textbf{Question \ref{q4}}, both $\left<x\right>,\left<p\right>$ now have an extra phase factor of $\phi$ in the time dependence (specificaly, the oscillation term), so for $\phi$ not being integer multiples of $2\pi$, the solution in \text{Question \ref{q4}} and here would be out of phase. 
        
        For $\phi=\frac{\pi}{2}$, we have $\left<x\right>=\frac{a}{2}-\frac{48a}{25\pi^2}\cos(5wt-\frac{\pi}{2})=\frac{a}{2}-\frac{48a}{25\pi^2}\sin(5wt)$, while $\left<p\right> = \frac{24\hbar}{5a}\sin(5wt-\frac{\pi}{2})=-\frac{24\hbar}{5a}\cos(5wt)$, so compared to \text{Question \ref{q4}} both expressions differ by a phase of $\frac{\pi}{2}$ (or swapping from $\sin$ to $\cos$).

        For $|phi=\pi$, we have $\left<x\right>=\frac{a}{2}-\frac{48a}{25\pi^2}\cos(5wt-\pi) - \frac{a}{2}+\frac{48a}{25\pi^2}\cos(5wt)$, and $\left<p\right>=\frac{24\hbar}{5a}\sin(5wt-\pi)=-\frac{24\hbar}{5a}\sin(5wt)$, so compared to \text{Question \ref{q4}} both expressions differ by a phase of $\pi$, which adds an extra negative sign in front of the time dependence (indicating that their expectation value having opposite dynamics).
    \end{itemize}
\end{proof}

\newpage

\section*{6}
\begin{ques}\label{q6}
Solve the time-independent Schrödinger equation with appropriate boundary conditions
for the “symmetric” infinite square well with
\[
V(x) =
\begin{cases}
0 & -\frac{a}{2} \le x \le \frac{a}{2} \\
\infty & \text{otherwise}
\end{cases}
\]
Determine the wavefunctions $\psi_n$ and their energies by computing the solutions in each
region and matching at the boundaries. What coordinate change can you do to bring your
solutions for the $\psi_n$ into the same form as the ones we found in lecture for the infinite
square well with walls at $x = 0, a$?
\end{ques}

\begin{proof}
    Given this solution, we'll again asume that for $x\notin [-\frac{a}{2},\frac{a}{2}]$, we have $\psi(x)=0$ (due to the fact that outside of the well, with infinite potential it's impossible for a particle to exist), and for $\psi(x)$ to be continuous, we also need $\psi(\frac{a}{2})=\psi(-\frac{a}{2})=0$ as boundary conditions.

    \hfil

    Which, inside the interval $[-\frac{a}{2},\frac{a}{2}]$ we have $\psi(x)$ satisfies the Time-independent Schrödinger equation, which is characterized as follow (with $V(x)=0$ in this interval):
    \begin{align}
        -\frac{\hbar^2}{2m}\frac{d^2\psi}{dx^2}+V\psi=E\psi\implies \frac{d^2\psi}{dx^2}+\frac{2mE}{\hbar^2}\psi=0
    \end{align} 
    We can assume that $E$ is real (by \textbf{Question \ref{q2} part 1}). Which, let $-\alpha^2=\frac{2mE}{\hbar^2}$ (or $\alpha=\pm\frac{i\sqrt{2mE}}{\hbar}$), the above equation bevomes $\frac{d^2\psi}{dx^2}-\alpha^2\psi=0$, hence has a general solution of $\psi(x)=Ae^{\alpha x}+Be^{-\alpha x}$.

    \hfil

    Now, there are three cases to consider:
    \begin{itemize}
        \item First, if $E<0$, then $\alpha$ can only be chosen as real number (since $-\alpha^2=\frac{2mE}{\hbar^2}<0$ due to the fact that $2,m,\hbar>0$, so $\alpha^2>0$). Hence the solution $\psi(x)=Ae^{\alpha x}+Be^{-\alpha x}$ are with real coefficients. Then, with the boundary condition, we have:
        \begin{align}
            \psi\left(\frac{a}{2}\right)=Ae^{\frac{a\alpha}{2}}+Be^{-\frac{a\alpha}{2}}=0\implies B=-Ae^{a\alpha}\\
            \psi\left(-\frac{a}{2}\right)=Ae^{-\frac{a\alpha}{2}}+Be^{\frac{a\alpha}{2}}=0\implies B=-Ae^{-a\alpha}
        \end{align}
        Hence, we get that $B=-Ae^{a\alpha}=-Ae^{-a\alpha}$, showing that $A(e^{a\alpha}-e^{-a\alpha})=0$. However, for the scenario to be nontrivial, we must have $a\neq 0$; and, since in this situation we derived $\alpha^2>0$, we also have $\alpha\neq 0$, hence $a\alpha\neq 0$. This shows that $e^{a\alpha}\neq e^0=1$, or $e^{a\alpha}-e^{-a\alpha}\neq 0$. Hence, for the equation to be true, one must have $A=0$. 

        Yet, this implies that $B=-Ae^{a\alpha}=0$ also, showing that $\psi(x)=0$, which is a contradiction to the interpretation that $|\psi(x)|^2$ is a probability distribution (since we derived $|\psi(x)|^2=0$), so this is not a valid case.

        \hfil

        \item Else if $E=0$, since the differential equation reduces to $\frac{d^2\psi}{dx^2}=0$, hence we get $\psi(x)=cx+d$ for some $c,d\in\CC$. Yet, for it to satisfy boundary conditions, we need $\psi(-\frac{a}{2})=-\frac{ac}{2}+d=0$ (or $d=\frac{ac}{2}$) and $\psi(\frac{a}{2})=\frac{ac}{2}+d=0$ (or $d=-\frac{ac}{2}$), hence $d=-d$, showing that $d=0$. Yet, we results in $\psi(x)=cx$, which in case for $\psi(\frac{a}{2})=\frac{ac}{2}=0$ (where $\frac{a}{2}\neq 0$ can be assumed for nontrivial scenario), then $c=0$, showing $\psi(x)=0$. Yet, this violates the fact that $|\psi(x)|^2$ is a probability distribution (since then $|\psi(x)|^2=0$), so this is again not a valid case.
        
        \hfil

        \item Finally, if $E> 0$, we have $\alpha$ being purely imaginary, then $\psi(x)$ can be further decomposed using Euler's Formula. Let $w = \frac{\sqrt{2mE}}{\hbar}> 0$, we have $\alpha=\pm iw$, so we get $\psi(x)=Ae^{iw x}+Be^{-iw x} = C \cos(wx+\phi)$ for some unknown $C\in\CC$ and $\psi\in\RR$. Then, based on the boundary condition, we get:
        \begin{align}
            0=\psi\left(-\frac{a}{2}\right)=C\cos\left(-w\frac{a}{2}+\phi\right),\quad 0=\psi\left(\frac{a}{2}\right)=C\cos\left(w\frac{a}{2}+\phi\right)
        \end{align}
        Hence, we get that $C\cos\left(w\frac{a}{2}+\phi\right)=C\cos\left(-w\frac{a}{2}+\phi+wa\right)=0$, showing that $wa = k\pi$ for some $k\in\NN$ (due to the assumption that $a,w>0$, so $aw>0$, hence $k>0$), which enforces the following case to be true for the energy:
        \begin{align}
            wa = a\frac{\sqrt{2mE}}{\hbar}=k\pi\implies E=\frac{1}{2m}\left(\frac{\hbar k\pi}{a}\right)^2
        \end{align}
        Hence, we again get that there are discrete.

        Finally, since we have the condition $wa=k\pi$ for some $k\in\NN$, then $\frac{wa}{2}=\frac{k}{2}\pi$. Then, since $C\cos(w\frac{a}{2}+\phi)=C\cos(\frac{k}{2}\pi+\phi)=0$ (together with the requirement that $\psi(x)\neq 0$, which implies $C\neq 0$), then we must have $\cos(\frac{k}{2}\pi+\phi)=0$, showing that $\phi+\frac{k}{2}\pi = +\frac{2\ell+1}{2}\pi$ for some $\ell\in\ZZ$, showing that $\phi = \frac{(2\ell+1)-k}{2}\pi$.

        Which, we ran into another two classifications:
        \begin{itemize}
            \item[1)] If $k$ is odd (or $k=2l+1$ for some $l\in\NN$), then WLOG one can choose $\ell:=l$, and use $\phi=0$ as solution. In this case we get the solution $\psi_k(x)=C\cos(w_kx)$, where $w_k=\frac{\sqrt{2mE_k}}{\hbar}$ and $E_k=\frac{1}{2m}\left(\frac{\hbar k\pi}{a}\right)^2$.
            \item[2)] Else if $k$ is even (or $k=2l$ for some $l\in\NN$), WLOG choose $\ell:=l$ again, we get $\phi=\frac{2l+1-2l}{2}\pi=\frac{\pi}{2}$. Then, the solution becomes $\psi_k(x)=C\cos(w_kx+\frac{\pi}{2})=-C\sin(w_kx) = C'\sin(w_kx)$, where $w_k$ and $E_k$ have the same definition as the previous case.
        \end{itemize}
    \end{itemize}
    So, under infinite square well scenario, we have an extra restriction that the separation constant $E>0$, and is restricted to discrete level $E_n=\frac{1}{2m}\left(\frac{\hbar n\pi}{a}\right)^2$. Given the ``frequency'' $w_n:=\frac{\sqrt{2mE_n}}{\hbar}=\frac{n\pi}{a}$, the general separable solution $\psi_n(x)$ of this Time-Independent Schrödinger equation (after normalization, with coefficient $\sqrt{\frac{2}{a}}$) is given as follow on the region $[-\frac{a}{2},\frac{a}{2}]$ (while outside of this interval is $0$):
    \begin{align}
        \psi_n(x)=\begin{cases}
            \sqrt{\frac{2}{a}}\cos\left(\frac{n\pi}{a}x\right) & n \textmd{ is odd}\\
            \sqrt{\frac{2}{a}}\sin\left(\frac{n\pi}{a}x\right) & n \textmd{ is even}
        \end{cases}
    \end{align}


    \hfil

    For coordinate change into the infinite square well with walls at $x=0,a$, one can simply perform a coordinate transformation and obtain $\psi_n(x-\frac{a}{2})$, then any $x_0\in [0,a]$ would have $x_0-\frac{a}{2}\in [-\frac{a}{2},\frac{a}{2}]$, which matches the domain of this symmetric square well.
\end{proof}

\newpage

\section*{7}
\begin{ques}\label{q7}
Consider again the “symmetric” infinite square well from the previous problem. Suppose
we measure the energy of a particle in this box, and we find the ground-state energy
\[
E_1 = \frac{\hbar^2\pi^2}{2ma^2}.
\]
We then know the particle is in the ground state with wavefunction $\psi_1(x)$,
which is the $n = 1$ wavefunction you found in the previous problem. Then we suddenly
pull the walls of the well out rapidly so that they are at $x = \pm 2a$ instead of $x = \pm a/2$; we do
it so rapidly that the state of the particle doesn’t change at that moment. Now, of course,
the particle is no longer in a state of definite energy in the new well; its wavefunction is no
longer a single separable solution of the Schrödinger equation with the larger box.

(a) Find the solutions to the time-independent Schrödinger equation (the $\psi_n$ and $E_n$) in
the new well with walls at $x = \pm 2a$.

(b) We measure the energy of the particle right after pulling the walls out to $x = \pm 2a$.
What is the most probable result of this measurement? What is the probability of
this result?

(c) What is the next-most probable result, and the probability of this result?

(d) What is the expectation value of the energy? Why should you have expected this
answer?

You probably think this is a crazy example, but it is a good analogy for what happens to
the potential experienced by the electron in tritium (an isotope of hydrogen whose nucleus
consists of one proton and two neutrons) when nuclear reactions convert the tritium into
${}^3$He (whose nucleus consists of two protons and one neutron).
\end{ques}

\begin{proof}

    \hfil

    \begin{itemize}
        \item[(a)] Here, if we define $a':=4a$, then then the square well with domain $[-2a,2a]$ becomes $[-\frac{a'}{2},\frac{a'}{2}]$. Hence, adapt the solution from \textbf{Question \ref{q6}}, we get that $E_n=\frac{1}{2m}\left(\frac{\hbar n\pi}{a'}\right)^2 = \frac{1}{32m}\left(\frac{\hbar n\pi}{a}\right)^2$, while the frequency $w_n=\frac{\sqrt{2m E_n}}{\hbar} = \frac{n\pi}{a'}=\frac{n\pi}{4a}$, and the general solution (with normalization constant $\sqrt{\frac{2}{a'}}=\frac{1}{\sqrt{2a}}$) of the new well is given by the follow in $[-2a,2a]$ (and $0$ everywhere else):
        \begin{align}
            \psi_n(x)=\begin{cases}
                \frac{1}{\sqrt{2a}}\cos\left(\frac{n\pi}{4a}x\right) & n \textmd{ is odd}\\
                \frac{1}{\sqrt{2a}}\sin\left(\frac{n\pi}{4a}x\right) & n \textmd{ is even}
            \end{cases}
        \end{align}

        \item[(b)] When the well instantly changes, we have the wave function still stays the same as the original constant $E_1=\frac{\hbar^2\pi^2}{2ma^2}$, which is with frequency $\frac{\sqrt{2m E_1}}{\hbar}=\frac{\pi}{a}$. Or, the ``new'' wave function of the particle is given by $\psi(x)=\sqrt{\frac{2}{a}}\cos\left(\frac{\pi}{a}x\right)$ on the interval $[-\frac{a}{2},\frac{a}{2}]$ and $0$ everywhere else, using the formula in \textbf{Question \ref{q6}} (with $n=1$, like the energy level).
        
        Which, using the new set of $\psi_n(x)$ given in part (a) (which in fact forms an orthonormal list, based on the properties of Fourier Series), then we get the following:
        \begin{align}
            \psi(x)=\sum_{n=1}^{\infty}c_n \psi_n(x),\quad \forall n\in\NN,\ c_n:=\int_{-2a}^{2a}\psi(x)\psi_n(x)^* dx
        \end{align}
        Doing the calculation, every $n\in\NN$ corresponds to the following constants. Which we'll use the substitution $u=\frac{\pi}{4a}x$, with differential $du=\frac{\pi}{4a}dx$ (Note: for $x\notin [-\frac{a}{2},\frac{a}{2}]$, $\psi(x)=0$, so we don't include it):
        \begin{align}
            n\textmd{ is odd:}\quad c_n &= \int_{-\frac{a}{2}}^{\frac{a}{2}}\sqrt{\frac{2}{a}}\cos\left(\frac{\pi}{a}x\right)\cdot \frac{1}{\sqrt{2a}} \cos\left(\frac{n\pi}{4a}x\right) dx = \frac{4}{\pi}\int_{-\frac{\pi}{8}}^{\frac{\pi}{8}}\cos\left(4u\right)\cos\left(n u\right)du\\
            &= \frac{2}{\pi}\int_{-\frac{\pi}{8}}^{\frac{\pi}{8}}\left(\cos((4+n)u)+\cos((4-n)u)\right)du \\
            &= \left(\frac{2}{(4+n)\pi}\sin((4+n)u)+\frac{2}{(4-n)\pi}\sin((4-n)u)\right)\bigg|_{-\frac{\pi}{8}}^{\frac{\pi}{8}}\\
            &= \left(\frac{4}{(4+n)\pi}+\frac{4}{(4-n)\pi}\right)\cos\left(\frac{n\pi}{8}\right) = -\frac{32}{\pi(n^2-16)}\cos\left(\frac{n\pi}{8}\right)
        \end{align}
        \begin{align}
            n\textmd{ is even:}\ c_n&=\int_{-\frac{a}{2}}^{\frac{a}{2}}\sqrt{\frac{2}{a}}\cos\left(\frac{\pi}{a}x\right)\frac{1}{\sqrt{2a}}\sin\left(\frac{n\pi}{4a}x\right) dx = 0
        \end{align}
        (Note: the second is $0$ due to the fact that $\sin$ is an odd function).

        Hence, we get the following decomposition of $\psi(x)$:
        \begin{align}
            \psi(x)&=\sum_{n=1}^{\infty}c_n\psi_n(x)=\sum_{k=0}^{\infty}c_{2k+1}\psi_{2k+1}(x) = \frac{-32}{\pi\sqrt{2a}}\sum_{k=0}^{\infty}\frac{\cos\left(\frac{(2k+1)\pi}{8}\right)}{(2k+1)^2-16}\cos\left(\frac{(2k+1)\pi}{4a}x\right)
        \end{align}
        Which, since each state $\psi_n(x)$ (specifically, their corresponding energy $E_n$) has a probability of being measured stated as $|c_n|^2$, to find the most probable energy state, we need to maximize $|c_n|^2$. For $n$ even, since $c_n=0$, it's trivial; for $n$ odd instead, since $|c_n|^2=\cos^2(\frac{n\pi}{8})\cdot\frac{1}{(n^2-16)^2}\cdot \left(\frac{32}{\pi}\right)^2$, it suffices to maximize $\frac{\cos^2(n\pi/8)}{(n^2-16)^2}$.

        Use the function $\frac{\cos^2(\pi x/8)}{(x^2-16)^2}$, numerically the function is constantly increasing for $x<0$ and constantly decreasing for $x>0$, hence the positive integer with the largest such value would be $n=x=1$. 

        Hence, after measuring the most probable result of this measureemnt still turns out to be the ``new'' first energy state $E_1=\frac{1}{32m}\left(\frac{\hbar\pi}{a}\right)^2$ (using the new energy provided in part (a)), and the corresponding probability of measuring energy $E_1$ is $\PP(E_1)=|c_1|^2 = \cos^2(\pi/8)\cdot\frac{1}{225}\cdot\left(\frac{32}{\pi}\right)^2 \approx 0.3936$.

        \hfil

        \item[(c)] Based on the result in part (b), since the ``corresponding function'' of probability $|c_n|^2$ of odd $n$ (after swapping $n$ with $x$) is constantly decreasing for $x>0$, then the next most probable result would be $n=3$ (the next odd positive integer after $1$). Which, the next most probable result of energy measurement is $E_3 = \frac{1}{32m}\left(\frac{\hbar 3\pi}{a}\right)^2$, and the corresponding probability of measuring energy $E_3$ is $\PP(E_3)=|c_3|^2=\cos^2(3\pi/8)\cdot\frac{1}{49}\cdot\left(\frac{32}{\pi}\right)^2\approx 0.3101$.
        
        \hfil

        \item[(d)] The expectation value of energy, $\left<\hat{H}\right> = \sum_{n=1}^{\infty}|c_n|^2 E_n = \left(\frac{32}{\pi}\right)^2\sum_{k=0}^{\infty}\frac{\cos^2(n\pi/8)}{(n^2-16)^2}\cdot \frac{1}{32m}\left(\frac{\hbar n\pi}{a}\right)^2 = \frac{32\hbar^2}{ma^2}\sum_{k=0}^{\infty}\frac{n^2\cos^2(n\pi/8)}{(n^2-16)^2}$, where $n=(2k+1)$ in each summand. Numerically, the infinite series in the term $\sum_{k=0}^{\infty}\frac{(2k+1)^2\cos^2((2k+1)\pi/8)}{((2k+1)^2-16)^2} = \frac{\pi^2}{64}$, which eventually have the expectation value of energy $\left<\hat{H}\right> = \frac{32\hbar^2}{ma^2}\cdot \frac{\pi^2}{64}=\frac{\hbar^2\pi^2}{2ma^2}$, which is the initial energy after the measurement of the particle's energy (which happened before the well is changed).
        
        This value is expected, since the particle constantly stays in a potential well without any other interaction, so there are no external interaction (or external energy being added to the system). Even after changing the width of the potential well, the particle still doesn't interact with the external environment, hence we're not expecting any external energy to be added to the system. Hence, the expected energy value should still stay the same.
   \end{itemize}
\end{proof}

\newpage

\end{document}
