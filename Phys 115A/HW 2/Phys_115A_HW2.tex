\documentclass{article}
\usepackage{../../Self_Style}

\title{Phys 115A HW 2}
\author{Zih-Yu Hsieh}
\date{\today}

\begin{document}
\maketitle

\section*{1 (4 ND)}
\begin{ques}\label{q1}
To build a little more experience with separation of variables, consider the heat equation
in one spatial dimension:
\[
\frac{1}{k}\frac{\partial u(x, t)}{\partial t} = \frac{\partial^2 u(x, t)}{\partial x^2}
\]
where $u(x, t)$ corresponds to the temperature of an object with thermal conductivity $k$.

1. Using separation of variables, $u(x, t) = u_x(x)u_t(t)$, and separation constant $-\alpha^2$, separate the heat equation into ordinary differential equations for $u_t$ and $u_x$.

2. Find the general solution to the differential equation satisfied by $u_x(x)$.

3. Find the general solution to the differential equation satisfied by $u_t(t)$.

4. Write down the general solution $u(x, t)$ to the heat equation.
\end{ques}

\begin{proof}

    \hfil

    \begin{enumerate}
        \item Given the separation of variables $u(x,t)=u_x(x)u_t(t)$, plug into the formula we get the following:
        \begin{align}
            \frac{1}{k}\frac{\partial}{\partial t}(u_x(x)u_t(t))=\frac{\partial^2}{\partial x^2}(u_x(x)u_t(t))&\implies \frac{1}{k}\frac{d u_t}{dt}u_x(x)=\frac{d^2u_x}{dx^2}u_t(t)\\
            &\implies \frac{1}{k}\cdot \frac{1}{u_t(t)}\frac{du_t}{dt}=\frac{1}{u_x(x)}\frac{d^2u_x}{dx^2}
        \end{align}
        Notice that the line aboe have the left side being dependent only on $t$, while the right side is solely dependent on $x$, hence the whole equation must be constant. With the separation constant given by $-\alpha^2$, we get the following two ordinary differential equations:
        \begin{align}
            &\frac{1}{k}\cdot \frac{1}{u_t(t)}\frac{du_t}{dt}=\frac{1}{u_x(x)}\frac{d^2u_x}{dx^2}=-\alpha^2\\
            &\implies \begin{cases}
                \frac{du_t}{dt}=-k\alpha^2 u_t(t)\\
                \frac{d^2u_x}{dx^2}=-\alpha^2u_x(x)
            \end{cases}
        \end{align}

        \hfil

        \item For $u_x(x)$, since it satisfies $\frac{d^2u_x}{dx^2}+\alpha^2u_x(x)=0$ (or can rewrite as $\frac{d^2 u_x}{dx^2}-(i\alpha)^2 u_x(x)=0$), then the general solution of this is given by $Ae^{i\alpha x}+Be^{-i\alpha x}$, where $A,B\in\CC$ are arbitrary constants (depending on the initial condition).
        
        (Note: If want real solution instead, it depends on the constant $\alpha$, so here we'll provide the most general form).

        \hfil
        
        \item For $u_t(t)$, since it satisfies $\frac{du_t}{dt}=-k\alpha^2 u_t(t)$, then the general solution  $u_t(t)=Ke^{-k\alpha^2 t}$, where $K\in\CC$ is an arbitrary constant (depending on the initial condition).
        
        (Note: If want real solution just take the arbitrary constants in $\RR$ instead).

        \hfil
        
        \item If plug
    \end{enumerate}
\end{proof}

\newpage

\section*{2}
\begin{ques}\label{q2}
Prove the following three theorems:

1. For normalizable solutions, the separation constant $E$ must be real. Hint: Write $E$
(in Griffiths eq. 2.7, i.e. $\Psi(x, t) = \psi(x)e^{-iEt/\hbar}$) as $E_0 + i\Gamma$, with $E_0$ and $\Gamma$ real, and
show that $\Gamma$ must be zero if Griffiths eq. 1.20 (i.e.
\[
\int_{-\infty}^{\infty} |\Psi(x, t)|^2 dx = 1
\]
is to hold for all $t$.

2. The time-independent wavefunction $\psi(x)$ can always be taken to be real (unlike
$\Psi(x, t)$, which is necessarily complex). This doesn’t mean that every solution to the
time-independent Schrödinger equation is real; what it says is that if you’ve got one
that is not, it can always be expressed as a linear combination of solutions (with the
same energy) that are. So you might as well stick to $\psi$’s that are real. Hint: if $\psi(x)$
satisfies Griffiths eq. 2.5 (i.e.
\[
-\frac{\hbar^2}{2m}\frac{d^2\psi}{dx^2} + V\psi = E\psi
\]
for a given $E$, so too does its complex conjugate, and hence also the real linear combinations
$\psi+\psi^*$ and $i(\psi-\psi^*)$.

3. If $V(x)$ is an even function (that is, $V(-x) = V(x)$) then $\psi(x)$ can also be taken to
be either even or odd. Hint: If $\psi(x)$ satisfies Griffiths eq. 2.5, for a given $E$, then so
too does $\psi(-x)$, and hence also the even and odd linear combinations $\psi(x)\pm\psi(-x)$.
\end{ques}

\begin{proof}

    \hfil

    \begin{enumerate}
        \item Suppose $\Psi(x,t)$ is a separable solution to the Schrödinger equation that is a scaled version of a wave function (so it's normalizable), then since $\Psi(x,t)=\psi(x)e^{-iE t/\hbar}$ for some constant $E\in\CC$, then $E=E_0+i\Gamma$ for some $E_0,\Gamma\in\RR$. Now, suppose the solution $\Psi$ itself is normalized, we have the following:
        \begin{align}
            1&=\int_{-\infty}^{\infty}|\Psi(x,t)|^2 dx=\int_{-\infty}^{\infty}\left(\psi(x)e^{-i(E_0+i\Gamma)t/\hbar}\right)\left(\psi^*(x)\left(e^{-i(E_0+i\Gamma)t/\hbar}\right)^*\right)dx\\
            &= \int_{-\infty}^{\infty}|\psi(x)|^2\cdot e^{2\Gamma t/\hbar}dx = e^{2\Gamma t/\hbar}\int_{-\infty}^{\infty}|\psi(x)|^2 dx
        \end{align}
        Since $\int_{-\infty}^{\infty}|\psi(x)|^2 dx$ must stay as constant for all $t\in\RR$, for the whole expression to be $1$ for all $t\in\RR$, one must have $e^{2\Gamma t/\hbar}$ being a constant also. Hence we must have $\Gamma=0$, showing that $E=E_0+i\Gamma=E_0\in\RR$.

        \hfil

        \item Suppose $\psi(x)$ satisfies the time independent Schrödinger equation, we have the following:
        \begin{align}
            -\frac{\hbar^2}{2m}\frac{d^2\psi}{dx^2}+V\psi=E\psi
        \end{align}
        Where both $V,E\in\RR$. Hence, if taken the complex conjugation, we get:
        \begin{align}
            \left(-\frac{\hbar^2}{2m}\frac{d^2\psi}{dx^2}\right)^*+(V\psi)^*=(E\psi)^*\implies -\frac{\hbar^2}{2m}\frac{d^2\psi^*}{dx^2}+V\psi^*=E\psi^*
        \end{align}
        Hence, $\psi^*$ is also a solution to the Time-independent Schrödinger equation. As a result, since $\psi(x)=\Real(\psi)(x)+i \Imag(\psi)(x)$, where $\Real(\psi)(x)=\frac{1}{2}(\psi(x)+\psi^*(x))$, and $\Imag(\psi)(x)=\frac{1}{2i}(\psi(x)-\psi^*(x))$, then since linear combinations of solutions to a differential equation is still a solution (and the fact that $\psi:\RR\rightarrow \CC\cong \RR^2$ is differentiable, one must have both its real and imaginary parts being differentiable also), then $\Real(\psi)(x),\Imag(\psi)(x)$ are both solutions to the Time-Independent Schrödinger equation. Hence, with these two components being real, $\psi(x)$ can be written as linear combinations of real solutions.

        \hfil

        \item Given that $V(x)$ is even, then $V(x)=V(-x)$. Now, if we consider $\psi(-x)$ instead, then we have $\frac{d}{dx}\psi(-x)=-\psi'(-x)$, and $\frac{d^2}{dx^2}\psi(-x)=\frac{d}{dx}\left(-\psi'(-x)\right) = \psi''(-x)$. Hence, since $\psi(x)$ is a solution to the Time-Independent Schrödinger equation by assumption, we have $-\frac{\hbar^2}{2m}\psi''(x)+V(x)\psi(x)=E\psi(x)$, then plug in $u=-x$ instead, we get $-\frac{\hbar^2}{2m}\psi''(-x)+V(-x)\psi(-x)=E\psi(-x)$, which implies the following:
        \begin{align}
            -\frac{\hbar^2}{2m}\frac{d^2}{dx^2}(\psi(-x))+V(x)\psi(-x)=E\psi(-x)
        \end{align}
        Hence, $\psi(-x)$ is also a solution. 

        Then, as a result $\frac{\psi(x)+\psi(-x)}{2}$ and $\frac{\psi(x)-\psi(-x)}{2}$ are both solutions to the time-independent Schrödinger equation, where the first one is even, the latter one is odd, and $\psi(x)=\frac{\psi(x)+\psi(-x)}{2}+\frac{\psi(x)-\psi(-x)}{2}$. Hence, any solution $\psi(x)$ can be expressed as linear combinations of odd and even functions.
    \end{enumerate}
\end{proof}

\newpage

\section*{3  (ND)}
\begin{ques}\label{q3}
Show that $E$ must exceed the minimum value of $V(x)$, for every normalizable solution to
the time-independent Schrödinger equation. What is the classical analog to this statement?

Hint: Rewrite Griffiths eq. 2.5 (i.e.
\[
-\frac{\hbar^2}{2m}\frac{d^2\psi}{dx^2} + V\psi = E\psi
\]
in the form
\[
\frac{d^2\psi}{dx^2} = \frac{2m}{\hbar^2}[V(x) - E]\psi;
\]
if $E < V_{\min}$, then $\psi$ and its second derivative always have the same sign – argue that such
a function cannot be normalized.
\end{ques}

\begin{proof}
    Suppose the contrary that $E< V_{\min}$, then rewrite the Time-Independent Schrödinger equation, we get:
    \begin{align}
        \frac{d^2\psi}{dx^2} = \frac{2m}{\hbar^2}(V(x) - E)\psi
    \end{align}
    Where, since $E<V_{\min}$, then let $\epsilon=V_{min}-E>0$, we have 
\end{proof}

\newpage

\section*{4}
\begin{ques}\label{q4}
You prepare a particle in the infinite square well (with walls at $x = 0, a$) in an initial state
described by a linear combination of two stationary states,
\[
\Psi(x, 0) = A[\psi_2(x) + \psi_3(x)]
\]
where $\psi_2, \psi_3$ are given by Griffiths eq. 2.31 (i.e. $\psi_n(x) = \sqrt{\frac{2}{a}}\sin\left(\frac{n\pi x}{a}\right)$) with $n = 2, 3$
respectively.

(a) Normalize $\Psi(x, 0)$, i.e., find $A$.

(b) Find $\Psi(x, t)$ and $|\Psi(x, t)|^2$. Express the latter as a sinusoidal function of time in
terms of the variable $\omega = \pi^2\hbar / 2ma^2$.

(c) Compute $\langle x \rangle$. Is this an interesting function of time? Note: the integral is painful,
but please work your way through it by hand and show your work. Integration by
parts and trigonometric identities (including one that Griffiths uses while proving
orthogonality of the stationary states) are both your friends!

(d) Compute $\langle p \rangle$.

(e) If you measured the energy of the particle, what values might you get, and what is
the probability of getting them? What is the expectation value of $H$, and how does
this relate to the energies $E_2$ and $E_3$ of $\psi_2$ and $\psi_3$?
\end{ques}

\begin{proof}
\end{proof}

\newpage

\begin{ques}\label{q5}
We have emphasized that overall phases of the wave function are irrelevant, since they
cancel out of physical quantities. But the relative phase matters! Imagine we change the
phase in the previous problem so that our initial state is
\[
\Psi(x, 0) = A[\psi_2(x) + e^{i\phi}\psi_3(x)]
\]
for some real constant $\phi$.

(a) Find $\Psi(x, t)$ and $|\Psi(x, t)|^2$.

(b) Find $\langle x \rangle$.

(c) Find $\langle p \rangle$.

(d) Discuss how these results differ from the case of $\phi = 0$. Consider especially the cases
$\phi = \pi/2$ and $\phi = \pi$.
\end{ques}

\begin{proof}
\end{proof}

\newpage

\begin{ques}\label{q6}
Solve the time-independent Schrödinger equation with appropriate boundary conditions
for the “symmetric” infinite square well with
\[
V(x) =
\begin{cases}
0 & -\frac{a}{2} \le x \le \frac{a}{2} \\
\infty & \text{otherwise}
\end{cases}
\]
Determine the wavefunctions $\psi_n$ and their energies by computing the solutions in each
region and matching at the boundaries. What coordinate change can you do to bring your
solutions for the $\psi_n$ into the same form as the ones we found in lecture for the infinite
square well with walls at $x = 0, a$?
\end{ques}

\begin{proof}
\end{proof}

\newpage

\begin{ques}\label{q7}
Consider again the “symmetric” infinite square well from the previous problem. Suppose
we measure the energy of a particle in this box, and we find the ground-state energy
\[
E_1 = \frac{\hbar^2\pi^2}{2ma^2}.
\]
We then know the particle is in the ground state with wavefunction $\psi_1(x)$,
which is the $n = 1$ wavefunction you found in the previous problem. Then we suddenly
pull the walls of the well out rapidly so that they are at $x = \pm 2a$ instead of $x = \pm a/2$; we do
it so rapidly that the state of the particle doesn’t change at that moment. Now, of course,
the particle is no longer in a state of definite energy in the new well; its wavefunction is no
longer a single separable solution of the Schrödinger equation with the larger box.

(a) Find the solutions to the time-independent Schrödinger equation (the $\psi_n$ and $E_n$) in
the new well with walls at $x = \pm 2a$.

(b) We measure the energy of the particle right after pulling the walls out to $x = \pm 2a$.
What is the most probable result of this measurement? What is the probability of
this result?

(c) What is the next-most probable result, and the probability of this result?

(d) What is the expectation value of the energy? Why should you have expected this
answer?

You probably think this is a crazy example, but it is a good analogy for what happens to
the potential experienced by the electron in tritium (an isotope of hydrogen whose nucleus
consists of one proton and two neutrons) when nuclear reactions convert the tritium into
${}^3$He (whose nucleus consists of two protons and one neutron).
\end{ques}

\begin{proof}
\end{proof}

\newpage

\end{document}
